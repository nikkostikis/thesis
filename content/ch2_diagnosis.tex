\chapter{Diagnosis of Parkinson's Disease}
\label{ch:diagnosis}
\pagestyle{fancy}
\fancyhf{}
\fancyhead[OC]{\leftmark}
\fancyhead[EC]{\rightmark}
\cfoot{\thepage}

%%%%%%%%%%%%%%%%%%%%%%%%%%%%%%%%%%%%%%%%%%%%%%%%%%%%%%%%%%%
%%%%%%%%%%%%%%%%%%%%%%%%%%%%%%%%%%%%%%%%%%%%%%%%%%%%%%%%%%%

\section{Clinical Scale-Based Diagnosis}
\label{sec:scaleBased}
The \gls{PD} clinical evaluation mainstay is the face-to-face scale-based symptoms' assessment by an experienced physician. \gls{PD} symptoms, both motor and cognitive are frequently present in many other conditions. Apart from drug-induced parkinsonism there are also other diseases with similar cardinal symptoms, such as Progressive Supranuclear Palsy (\gls{PSP}), Essential Tremor, Lewey-body Dementia, Amyotrophic Lateral Sclerosis, Huntingtons Disease and Multiple System Atrophy. Although for most of them the typical cases can easily be distinguished, there are cases where only an experienced expert could differentiate accurately and definitively. Apart from the diagnosis, each \gls{PD} patient follows a unique neurodegeneration path and responds variably to the prescribed medication. That is the reason why the rich face-to-face scale-based examination has prevailed until today as the most common and reliable method of diagnosis and assessment regarding \gls{PD}. 

%%%%%%%%%%%%%%%%%%%%%%%%%%%%%%%%%%%%%%%%%%%%%%%%%%%%%%%%%%%

\subsection{UPDRS}
\label{subsec:updrs}
The most widely used clinical method for quantifying the symptoms of \gls{PD} is the Unified Parkinson's Disease Rating Scale, (\gls{UPDRS}) (Ebersbach et al, 2006). Introduced in 1987 and revised following a proposal from the Movement Disorder Society Task Force on Rating Scales for Parkinson's Disease (Goetz et al, 2008), the scale does not require any equipment and involves observing the patient in various postures and ``standardized'' movements and tasks, and ``grading'' their performance on a range of 0-4, 0 being normal, 1 slight, 2 mild, 3 moderate and 4 severe.

Although the scale uses numeric values the distances between these values (0-4) are not equal. Each part of the rating is a rank order measure and although, for example, a rating 4 is greater than 2, it is not twice as severe (Perlmutter, 2010). The scale in its \gls{MDS} revision consists of four parts:
\begin{enumerate}
\item Nonmotor experiences of daily living (nM-EDL),
\item Motor experiences of daily living (M-EDL),
\item Motor examination,
\item Motor complications
\end{enumerate}

Parts 1, 2 and 4 contain questions the physician puts to the patient, their caregiver or both of them. In part 1A probing questions, must be used to determine the correct coding. For these questions the physician must avoid using \gls{UPDRS} terminology and instead try to infer the appropriate rating. In order to decide how to evaluate the answer on the scale of 0 to 4 the physician has to follow the process described in Figure ~\ref{fig:UPDRS1A}. Physicians are encouraged by the \gls{MDS} to work up and down the options with the patient to define the best fitting response.

%% graph will be placed here
\newcommand{\rows}[2]{% #1 = rows, #2 = style
\foreach \r in {#1} {%
  \globaldefs=1\relax
  \tikzset{row \r/.style={#2}}
}%
}
\newcommand{\cols}[2]{% #1 = columns, #2 = style
\foreach \r in {#1} {%
  \globaldefs=1\relax
  \tikzset{column \r/.style={#2}}
}%
}
\tikzset{
  centered/.style = {align=center, anchor=center},
       box/.style = {font=\sffamily, rectangle, rounded corners, 
       				centered, fill=white, draw=black!50, thin, text width=5cm, minimum height=2cm},
  boxArrow/.style = {font=\sffamily, single arrow, single arrow head extend=0ex,
  					centered, fill=black!30, text=white, text width=2.5cm, 
                    minimum height=4cm, minimum width=2cm, inner sep=0.1cm},
 vertArrow/.style = {->,>=Triangle, centered, line width=0.9cm, black!30},
 arrowText/.style = {font=\sffamily, midway, black, fill=white, centered}
}
\begin{figure}
\centering
\begin{tikzpicture}[every node/.style={outer sep=0, inner sep=0.15cm}]
  \cols{1,3}{nodes={box}}
  \cols{2}{nodes={boxArrow}}
  \matrix (m)
    [matrix of nodes, column sep=0.3cm, row sep=1.8cm]
    {
      \textit{to patient:} Is this item normal for you?  
      & \textit{patient:} ``Yes''
      & mark (0) \textbf{Normal} \\
      
      \textit{to patient:} Consider mild as a reference point and then compare with slight  
      & \textit{patient:} ``Yes, slight is closer''
      & confirm and mark (1) \textbf{Slight} \\
      
      \textit{to patient:} Consider moderate to see if this answer fits better  
      & \textit{patient:} ``No, moderate is too severe''
      & confirm and mark (2) \textbf{Mild} \\

	  \textit{to patient:} Consider severe to see if this answer fits better  
      & \textit{patient:} ``No, severe is too severe''
      & confirm and mark (3) \textbf{Moderate} \\

      confirm and mark (4) \textbf{Severe} & &\\
    };
  \begin{scope}[on background layer]
    \draw[vertArrow] (m-1-1) -- (m-2-1) node[arrowText]{\textit{patient:} ``No, I have problems''};
    \draw[vertArrow] (m-2-1) -- (m-3-1) node[arrowText]{if mild is closer than slight};
    \draw[vertArrow] (m-3-1) -- (m-4-1) node[arrowText]{if moderate is closer than mild};
    \draw[vertArrow] (m-4-1) -- (m-5-1) node[arrowText]{\textit{patient:} ``Yes, severe is closest''};
  \end{scope}
\end{tikzpicture}
\caption{Algorithm to define UPDRS rating for part 1A. Adapted from Goetz et al, 2008} \label{fig:UPDRS1A}
\end{figure}

In part 1 the questions pertain to issues like cognitive impairment, hallucinations and psychosis, depressed mood, anxious mood, and apathy. There are also questions which regard sleep problems, pain, urinary problems, constipation problems, light headedness on standing, and fatigue. Part 2 questions regard speech, saliva and drooling, chewing and swallowing, eating, dressing, handwriting, tremor (as the patient experiences it), turning in bed, getting out of bed, a car, or a deep chair, walking and freezing.

As already mentioned, part 4 is also filled out by the physician in collaboration with the patient and his caregiver. Of course, the physician must take into consideration the examination he conducted for part 3, which will be described in the following paragraph. In part 4 the questions aim to assess two motor complications, dyskinesias and motor fluctuations, including OFF-state dystonia. As already discussed in \ref{sec:dyskinesia}, motor fluctuations happen when there are periods of the day with poor or absent motor response to medication, also called OFF-state, alternating with periods of improved motor function and good response to medication, the ON-state. The patient is encouraged to recall how much time he spends in the OFF-state, what is the impact of dyskinesia in his life and how complex and prevalent his motor fluctuations are.

Although information from parts 1, 2 and 4 of the \gls{UPDRS} questionnaire is relevant and important, the most decisive part of the clinical assessment is the protocol the physician follows to complete part 3, the motor examination. As described in (Goetz et al, 2008), for this part of the assessment the physician instructs the patient to assume specific positions and execute particular tasks, while the expert observes his performance. This includes:
\begin{itemize}
\item Engaging in conversation, to evaluate speech.
\item Sitting at rest without talking, to observe facial expressions.
\item Allowing the physician to manipulate his major joins like elbows, neck, shoulders, knees and ankles, without and with activation (Froment's maneuver), for rigidity. The neck must be tested separately, the wrist and elbows simultaneously, as well as the hip and knee. The activation could be some kind of task like tapping fingers, fist opening and closing, or heel tapping using the limb not being tested, to promote the exposure of rigidity possibly affecting the limb being tested (Broussolle et al, 2007). 
\item Tapping his fingers using his index and thumb, and open and close his fists as quickly and as big as possible to observe the motion's speed and amplitude, hesitation or halt. 
\item Sitting in a deep chair and try to get up while his hands are crossed in front of his chest to observe how successfully the task was performed, and the fluidity of the motion, any hesitation or help required. After the patient gets up the physician must observe his posture. 
\item Sitting in a chair, get up, walk a few meters, perform a U-turn, walk back to the chair, reposition himself in front of the chair and sit again. In this test, termed  Timed Up and Go (\gls{TUG}), balance and mobility are assessed.
\item Walking to and away from the physician to observe stride amplitude, speed, height of foot lift, and arm swing (akinesia).
\item Standing still while the physician will push and pull him by the shoulders abruptly to observe his response. Abnormal response would include taking more than two steps backwards or forwards, indicating postural instability, as well as not moving in a timely manner, indicating freezing of gait and hesitation. 
\item Standing or sitting with his back straight and his arms extended in front of his chest with the palm facing the ground. The fingers should not touch each other nor should they be stretched. This position allows the physician to assess the patient's postural tremor of the hands, first by identifying its presence and second by estimating its frequency and amplitude. This position should be maintained and observed for more than 10 seconds to accommodate for re-emergent tremor, i.e., tremor appearing about 5 seconds after the position has been assumed, which is typical for \gls{PD} patients (Chen and Swope, 2003). 
\item Sitting comfortably with feet touching the floor and hands resting palms-down on the arms of a chair specifically and not on the patient's lap, to observe the presence and estimate the frequency and amplitude of rest tremor in the hands. This position also allows the physician to observe rest tremor in the legs, lips and jaw.
\end{itemize}

The \gls{UPDRS} was not developed to be a screening tool but mainly a progression tracking tool (Perlmutter, 2010). Its use can be compared to an uncalibrated scale. Such a scale couldn't be used to measure the exact weight but it could be valuable in tracking whether one gains or loses weight and by what percentage of his initial state. The grading system of the \gls{UPDRS} is rather coarse grained, only allowing for integer values, and with a limited range. However, if the measurements are taken at carefully planned moments, i.e., pre-medication, immediately, and a couple of hours post-medication, when the treatment has worn off, by the same expert physician, the progression of the disease and the response to the treatment regimen can be accurately assessed and used as input to a potential re-adjustment of the symptoms' alleviation approach. Having said that, the \gls{UPDRS} has been used for massive screenings of potential parkinsonism. In (Racette et al, 2006) a modified version of part 3 (motor examination) was used with blinded videotaped ratings; that study found that a high score in this modified part 3 of the scale had 100\% sensitivity and 81\% specificity in recognizing parkinsonism in more than 2000 people. This means that all participants in the study who suffered from parkinsonism scored high under the \gls{UPDRS} criteria and were positively identified, whereas 19\% of people with no parkinsonism were falsely classified. 

%%%%%%%%%%%%%%%%%%%%%%%%%%%%%%%%%%%%%%%%%%%%%%%%%%%%%%%%%%%

\subsection{Other Rating Scales}
\label{subsec:otherScales}
There are also other rating protocols, besides the \gls{UPDRS}, including the Columbia University Rating Scale (\gls{CURS}) and the Northwestern University Disability Scale (\gls{NUDS}), which along with the \gls{UPDRS} are the most evaluated, valid and reliable ones (Ramaker et al, 2002). There are also the Schawb and England Activities of Daily Living (\gls{SE}-ADL) and the Hoehn and Yahr Staging of \gls{PD} (\gls{HY}), which are single item rating systems to primarily measure the overall effect of the disease on a patient. The \gls{HY} has 5 stages and supports half-value increments, whereas the \gls{SE} scale works with percentages, measuring from 0\%-100\% how independent a patient is in his daily life. When a physician assesses a patient he usually integrates the \gls{MDS}-\gls{UPDRS} with \gls{HY} and \gls{SE} scales (Perlmutter, 2010). Movement Disorders Society has a number of rating scales that can be applied complementary to the \gls{UPDRS} to assess patients' symptoms with a finer grain\footnote{http://www.movementdisorders.org/MDS/Education/Rating-Scales.htm}, such as the Modified Bradykinesia Rating Scale, the Non-Motor Symptoms Scale and the Unified Dyskinesia Rating Scale.
Cognitive assessment can be complemented with the Parkinson's Disease Questionnaires, \gls{PDQ}-39 and \gls{PDQ}-8\footnote{https://innovation.ox.ac.uk/outcome-measures/parkinsons-disease-questionnaire-pdq-39-pdq-8/} (not to be confused with PDQ-4 which is a personality diagnostic questionnaire). These are self-completion tests consisting of questions on the following topics:

\begin{itemize}
\item mobility
\item activities of daily living
\item emotional well-being
\item stigma
\item social support
\item cognitions
\item communications
\item bodily discomfort
\end{itemize}

Patients are required to answer on the frequency of certain events happening in their daily living. Frequency can be defined as never, occasionally, sometimes, often, always or cannot do. In the \gls{PDQ}-39 there are multiple events for each of the aforementioned topics, whereas the \gls{PDQ}-8 is a shorter version consisting of only one event per each of the 8 topics.

Cognitive Assessment can also be provided by the Montreal Cognitive Assessment (\gls{MoCA})\footnote{http://www.mocatest.org} tool, which is generally used to assess attention and concentration, cognitive functions, memory, language, visuoconstructural skills, conceptual thinking, calculations and orientation. The test conductor asks the patient to fulfill a set of predefined tasks, assigning points for the extent and correctness of completion. The highest total score is 30 points, with 26 or above being considered normal performance. Although the \gls{MoCA} is not \gls{PD} specific, it is a valid and sensitive testing instrument for the assessment of cognitive impairment manifested in \gls{PD} patients (Dalrymple-Alford et al, 2010). 

Finally, cognitive impairment and dementia could be assessed through the Mini Mental Parkinson (\gls{MMP}), a test derived from the Mini Mental State Examination (\gls{MMSE}), specifically adapted for \gls{PD} (Larner, 2012). Patients scoring under the cut-off are marked as requiring further investigation to ascertain the cause for their impairment. 

%%%%%%%%%%%%%%%%%%%%%%%%%%%%%%%%%%%%%%%%%%%%%%%%%%%%%%%%%%%
%%%%%%%%%%%%%%%%%%%%%%%%%%%%%%%%%%%%%%%%%%%%%%%%%%%%%%%%%%%

\section{Diagnosis Performance and Misdiagnosis}
\label{sec:misdiagnosis}
Although the face-to-face interaction between clinicians and patients afforded by the scales-based rating process is very ``rich'' in information, it is nevertheless a subjective exercise, and depends heavily on the expert's experience, knowledge, objectivity and accuracy.

All the aforementioned scales are based on clinical criteria, namely the cardinal \gls{PD} manifestations, both motor and non-motor. The careful application of diagnostic criteria, such as tremor, bradykinesia and rigidity derived from existing clinicopathologic studies can increase the positive predictive value of diagnosis to over 95\% (Hughes et al, 2002). Nevertheless, a purely clinical assessment of the disease is inevitably a subjective procedure, and although additional factors can be used to increase the certainty of diagnosis, attempting to maximize the specificity of the criteria can lead to a significant decrease in the diagnostic sensitivity, sometimes excluding as many as one-third of ``true'' cases (Hughes et al, 2001). In light of the above, most of the times response to medication and exclusionary symptoms must be taken under consideration.

Although the diagnosis of \gls{PD} is straightforward when patients suffer from the cardinal symptoms, differentiating primary from secondary or atypical parkinsonism can be a challenging task especially in the onset of the manifestations, when symptoms may overlap with other syndromes and \gls{PD} later signs are non-existent (Tolosa et al, 2006).

Misdiagnosis of \gls{PD} can happen for a variety of reasons. In (Tolosa et al, 2006) the authors report that in their study more than 25\% of \gls{PD} patients did not respond well to medication (i.e. had increased duration of OFF-state), whereas only 75\% of the \gls{PD} population responded well and where correctly identified. The physicians in that study misdiagnosed \gls{PD} for essential tremor, Alzheimer's disease and drug-induced i.e., secondary parkinsonism. In all these conditions the cardinal manifestations of \gls{PD} are also present, making it difficult to differentiate. 

%%%%%%%%%%%%%%%%%%%%%%%%%%%%%%%%%%%%%%%%%%%%%%%%%%%%%%%%%%%
%%%%%%%%%%%%%%%%%%%%%%%%%%%%%%%%%%%%%%%%%%%%%%%%%%%%%%%%%%%

\section{Post-Mortem Examination}
\label{sec:lewyBodies}
Post-mortem examination can confirm whether the patient was suffering from the disease by the presence of Lewy bodies in the dopaminergic cells. Lewy bodies are concentrations of proteins formed in parts of the brain where cells die. The reason they form is not clear; they could be the cause of death of the cells or a defensive mechanism of the human body to the actual cause of death of cells. Their presence, however, denotes some kind of neurodegenerative disease. When a patient experiences parkinsonism and the autopsy exposes Lewy bodies in the substantia nigra's dopaminergic cells, \gls{PD} is confirmed. Unfortunately Lewy bodies can only be identified during autopsy and cannot be relied upon for diagnosis and treatment.

%%%%%%%%%%%%%%%%%%%%%%%%%%%%%%%%%%%%%%%%%%%%%%%%%%%%%%%%%%%
%%%%%%%%%%%%%%%%%%%%%%%%%%%%%%%%%%%%%%%%%%%%%%%%%%%%%%%%%%%

\section{Differential Diagnosis}
\label{sec:differential}
According to the United Kingdom's National Collaborating Centre for Chronic Conditions, features that can help physicians exclude \gls{PD} as the reason when a patient experiences the TRAP complex of symptoms include (National Collaborating Centre for Chronic Conditions, 2006):

\begin{itemize}
\item Repeated strokes with stepwise progression.
\item Repeated head injury.
\item Antipsychotic or dopamine-depleting drugs.
\item Definite encephalitis and/or oculogyric crises on no drug treatment.
\item Negative response to large doses of levodopa, when malabsorption is excluded.
\item Strictly unilateral features after 3 years.
\item Other neurological features: supranuclear gaze palsy, cerebellar signs, early severe autonomic involvement, Babinski sign, early severe dementia with disturbances of language, memory or praxis.
\item Exposure to known neurotoxin.
\item Presence of cerebral tumor or communicating hydrocephalus on neuroimaging.
\end{itemize}

\noindent
Observable features to support \gls{PD} diagnosis, apart from the \gls{UPDRS} part 3 clinical assessment, are:

\begin{itemize}
\item Unilateral onset.
\item Rest tremor present.
\item Progressive disorder.
\item Persistent asymmetry affecting the side of onset most.
\item Excellent response to levodopa.
\item Severe levodopa-induced dyskinesia.
\item Levodopa response for over 5 years.
\item Clinical course of over 10 years.
\end{itemize}

The UK National Collaborating Centre for Chronic Conditions suggests that at least three of the above need to be present to confirm a \gls{PD} diagnosis. 

%%%%%%%%%%%%%%%%%%%%%%%%%%%%%%%%%%%%%%%%%%%%%%%%%%%%%%%%%%%
%%%%%%%%%%%%%%%%%%%%%%%%%%%%%%%%%%%%%%%%%%%%%%%%%%%%%%%%%%%

\section{Neuroimaging Differential Diagnosis}
\label{sec:spectDat}
One can easily conclude that human expertise and adherence to a strict protocol are heavily relied upon when it comes to evaluating movement disorders. However, objective quantification and machine-based identification of pathological deviations is necessary to complement the physicians' instinct and expertise. Neuroimaging techniques are tools recently added to physicians' arsenal to properly diagnose \gls{PD} and differentiate from other conditions. In the last decade the advances in magnetic resonance and radiotracer-based imaging technology have increased the sensitivity and allowed for imaging of dopamine receptors in the striatum (Piccini and Brooks, 2006).

In single photon emission computed tomography (\gls{SPECT}), a gamma ray-emitting radioactive isotope is attached to a tracer molecule, which is given to the patient intravenously. This molecule (also called a ligand) labels neurons associated with dopamine re-uptake, which can be visualized in two-dimensional images. In controls and people with essential tremor, neuroleptic-induced parkinsonism or psychogenic parkinsonism the uptake is normal. The uptake is found reduced in those with \gls{PD} or Progressive Supranuclear Palsy (National Collaborating Centre for Chronic Conditions, 2006), another neurodegenerative disease, which shares cardinal manifestations with \gls{PD} and falls under the umbrella of Parkinson's Plus (see section \ref{sec:parkinsonism}). In 2011 the United States Food and Drug Administration (\gls{FDA}) approved the use of ioflupane iodine-123 (also called DaT-Scan), which is a specific tracer, which can help the observation of dopamine transporters through \gls{SPECT} and differentiate \gls{PD} from other conditions. In Europe DaTScan had been used since the early beginning of the century. Unfortunately it cannot help distinguish \gls{PD} from \gls{PSP}, where dopamine in the brain is also reduced.

Positron emission tomography (\gls{PET}) is similar to \gls{SPECT}. However, instead of focusing on the dopamine transporters, it focuses on the glucose metabolism in the brain. A positron-emitting radioactive isotope is attached to a tracer molecule and administered intravenously to the patient. The tracer is taken up by the dopaminergic neurons of the striatum (National Collaborating Centre for Chronic Conditions, 2006). \gls{PET} has been used to distinguish \gls{PD} from \gls{PSP} and other atypical parkinsonian syndromes (Segovia et al, 2015). The most important benefit these two tomographies offer is the examination of the function of the brain and not its anatomy. \gls{PD} patients do not have anatomical differences compared to healthy individuals. Structural magnetic resonance imaging (\gls{MRI}) provides two- and three-dimensional images of intracranial structures using high magnetic field strengths. In \gls{PD} this technique has been used to examine various structures known to be involved in the pathology of the condition, in the hope that it may prove of value in differential diagnosis (National Collaborating Centre for Chronic Conditions, 2006).

These imaging techniques cannot differentiate between conditions that affect the dopamine transporters, such as \gls{PD} and \gls{PSP}. However, they can exclude drug-induced or psychogenic parkinsonism and essential tremor. The interpretation of the imaging results must be performed in the context of the clinical symptoms of the patient. Along with the physician's one-on-one scale-based assessment the accuracy of the diagnosis can be increased. 

The downside of neuroimaging technology, like \gls{SPECT} and \gls{PET} scans is that it is extremely costly, in the range of a few thousand Euros (National Collaborating Centre for Chronic Conditions, 2006). For clinically indicated neuroimaging scans the cost is covered by national medical insurance plans in many European countries, in Canada and Australia. In the US it is covered by most private insurance companies. However, being ``free'' for the patient does not mean that these costs do not create inflationary pressure on insurance fees. 
