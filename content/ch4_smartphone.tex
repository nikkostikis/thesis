\chapter{A Smartphone-based Tool for Parkinsonian Hand Tremor Assessment}
\label{ch:smartphone}
\pagestyle{fancy}
\fancyhf{}
\fancyhead[OC]{\leftmark}
\fancyhead[EC]{\rightmark}
\cfoot{\thepage}

%%%%%%%%%%%%%%%%%%%%%%%%%%%%%%%%%%%%%%%%%%%%%%%%%%%%%%%%%%%
%%%%%%%%%%%%%%%%%%%%%%%%%%%%%%%%%%%%%%%%%%%%%%%%%%%%%%%%%%%

\section{General Approach}
\label{sec:genApproach}
As discussed in section ~\ref{subsec:smartphones}, smartphones are becoming useful platforms in the hands of researchers, clinicians and medical professionals, who by developing sophisticated software and protocols can incorporate these devices into various stages of clinical practice. Even cheap devices are equipped with embedded inertial sensors, bringing accelerometers and gyroscopes closer to every user. Combined with limitless connectivity and substantial processing power, they have the potential to become powerful quantification and tracking tools.

Incentivized by the research discussed in ~\ref{subsec:smartphones}, including the work of LeMoyne (2010) and Deneault (2013), we set out to investigate how the sensors embedded in a smartphone would perform in a clinical setting or even home environment, and if they could provide any useful methods for PD symptom quantification, identification and tracking. Our approach was to use a popular smartphone device (iPhone) and incorporate it in the same setting as the normal scale-based examination, namely UPDRS scoring, in a non-invasive, tireless manner. Having the six axes of innovation in mind (see ~\ref{subsec:axes}), our objectives revolved around cost-efficiency, availability, ease of use, quality of diagnosis, effective monitoring and biobanking. 



%More specifically, in the last decade the scientific community has been trying to find ways to assist medical experts and neurologists to overcome the problems of the \gls{PD} diagnostic protocol.

Our approach was supported by the following pillars: 
\begin{enumerate}
\item We focused on exploring the use of a ubiquitous device, such as a smartphone, with no additional software installed apart from its factory-installed operating system, and without the need to attach external hardware on it. 
\item We avoided making assumptions about the users, i.e. that they would have e-mail accounts or that they would be proficient enough to install and manage applications and configure complex settings. 
\item We explored solutions that would not necessarily require the presence of an expert to use, but would still produce replicable and accurate symptoms quantification. 
\item We opted for cloud based approaches that would ensure the persistence of the data and the availability of the results for post-processing and verification. 
\end{enumerate}


To validate our symptoms quantification approach we conducted two separate clinical trials: 
\begin{enumerate}
\item The first was a small-scale case control pilot clinical trial, which would serve as proof-of-concept for our approach (Kostikis et al, 2011). It consisted of 10 patients with idiopathic PD, defined as group PDtrial1, and 10 age-matched healthy volunteers, defined as group Htrial1. The data collected were post-processed using signal processing and statistically analyzed, to calculate quantifying metrics and establish their significance. 
\item The second was a larger case control clinical trial, were 23 patients with idiopathic PD, defined as group PDtrial2, and 20 age-matched healthy volunteers, defined as group Htrial2, were recruited. In a preliminary study of the results, we used the data of PDtrial2 to compare the quantification our tool calculated to the patients' UPDRS scores (Kostikis et al, 2014). Both the PDtrial2 and Htrial2 groups' data collected in the case control trial were processed and used to perform statistical analysis and build machine learning models to establish our tool's potential as a classification platform for PD patients (Kostikis et al, 2015). During the second clinical trial we conducted a small longitudinal trial to two idiopathic PD patients, defined as PDtrial2L. They were inpatients and were screened with our tool twice, once before, and once after medication. 
\end{enumerate}


In the following section we will describe the small-scale pilot trial and the preliminary results. Later on we will present the experimental setup for the second, larger case control clinical trial, the statistical analysis conducted and the machine-learning methods followed. Finally, we will discuss the results and implications derived from our smartphone-based PD symptoms quantification approach. 


\section{Pilot Smartphone Clinical Trial}
\label{sec:smartPilot}


this is a sample table
\begin{center}
 \begin{tabular}{||c c c c||} 
 \hline
 Col1 & Col2 & Col2 & Col3 \\ [0.5ex] 
 \hline\hline
 1 & 6 & 87837 & 787 \\ 
 \hline
 2 & 7 & 78 & 5415 \\
 \hline
 3 & 545 & 778 & 7507 \\
 \hline
 4 & 545 & 18744 & 7560 \\
 \hline
 5 & 88 & 788 & 6344 \\ [1ex] 
 \hline
\end{tabular}
\end{center}




%The goal of this work is to investigate the use of a smartphone-based tool for assessing PD induced hand tremor. Our approach involves using the phone’s embedded accelerometer and gyroscope sensors to quantify PD hand tremor. Operationally, the patient simply visits a web site [14] and takes up simple postures much like those used in standardized clinical exams, with the smartphone mounted on their hand. The data thus recorded can then be used to classify a subject as healthy or not, and to track the severity of the tremor in PD patients. 
%Preliminary versions of this work can be found in [15] and [16], where we presented proof-of-concept results for the tool discussed here. This paper substantially differs from and extends our previous work by i) including additional samples from healthy subjects which are age-matched to the PD patients used in our study, ii) including data on a small sample of patients off medication in order to quantitatively track the severity of their hand tremor, iii) exploring the correlation between our quantitative metrics results and the patients’ (subjective) clinical examination for all supported postures, and iv) using a machine learning feature classification approach to choose those metrics which are better at distinguishing between pathological and healthy signals, thus  increase our method’s accuracy. The accuracy in separating healthy from PD subjects attained in this work is on par with other works using smartphones’ accelerometers and short-duration data [12]. It is also very close to the sensitivity and specificity achieved in [17], where the authors used sensors of the SHIMMER platform to perform mobile gait analysis. There are works that achieve near 100\% accuracy but do so using day-long signals [18] which may not be practical in our setting. High accuracy (98.5\% sensitivity, 97.5\% specificity) is also achieved using smartphone accelerometer-based gait analysis, with a combination of tests and metrics [19]. 
%The proposed method has been implemented in the form of a website and is available for use on any smartphone with iOS or Android installed, without the need for any downloads or memory-consuming installations [14]. It can be offered as a web-service, so that developers can build their own applications around it and extend its functionality. Our approach does not require the presence of an expert or any kind of special equipment to conduct measurements. It transmits data in real time via TCP/IP, connecting the patient to his physician with no delays, and can benefit the research community by providing anonymized information on PD hand tremor profiles. 

%%%%%%%%%%%%%%%%%%%%%%%%%%%%%%%%%%%%%%%%%%%%%%%%%%%%%%%%%%%
%%%%%%%%%%%%%%%%%%%%%%%%%%%%%%%%%%%%%%%%%%%%%%%%%%%%%%%%%%%
\section{Experimental Setup}
\label{sec:expSetup}
%%%%%%%%%%%%%%%%%%%%%%%%%%%%%%%%%%%%%%%%%%%%%%%%%%%%%%%%%%%
\subsection{Volunteers}
\label{subsec:volunteers}
\subsubsection{Trial 1}
\label{subsec:volunteersTrial1}
%We recruited twenty-five PD patients from the outpatient clinic of the 1st Department of Neurology at the Aristotle University of Thessaloniki. They all agreed to participate after they were offered a detailed explanation of the study’s procedure and goals. All of them were right-handed, under L-DOPA treatment and suffering from PD for more than two years. During the study, two of them were hospitalized overnight so that they could be tested in the morning before they received their medication, to approximate de novo PD patients. Those two will be referred as our PDDN group, while the PD group comprises the other twenty-three (see Table I for patients’ information). The control group for the study, labeled as group H, contains twenty healthy volunteers, none of whom suffered from a movement disorder, hypertension or diabetes. They were screened for several health conditions which could exclude them from the study, such as hypertension or any movement disorder. They were also notified of the procedure and the purpose of the study before agreeing to participate. Grouping information on all participants of the study is provided in Table II.
%The ages of the two main groups were mean-tested with the non-parametric Mann-Whitney test and were found not to be statistically different at the 1\% significance level, therefore the groups can be considered age-matched.
\subsubsection{Trial 2}
\label{sebsec:volunteersTrial2}
%We recruited twenty-five PD patients from the outpatient clinic of the 1st Department of Neurology at the Aristotle University of Thessaloniki. They all agreed to participate after they were offered a detailed explanation of the study’s procedure and goals. All of them were right-handed, under L-DOPA treatment and suffering from PD for more than two years. During the study, two of them were hospitalized overnight so that they could be tested in the morning before they received their medication, to approximate de novo PD patients. Those two will be referred as our PDDN group, while the PD group comprises the other twenty-three (see Table I for patients’ information). The control group for the study, labeled as group H, contains twenty healthy volunteers, none of whom suffered from a movement disorder, hypertension or diabetes. They were screened for several health conditions which could exclude them from the study, such as hypertension or any movement disorder. They were also notified of the procedure and the purpose of the study before agreeing to participate. Grouping information on all participants of the study is provided in Table II.
%The ages of the two main groups were mean-tested with the non-parametric Mann-Whitney test and were found not to be statistically different at the 1\% significance level, therefore the groups can be considered age-matched.
%%%%%%%%%%%%%%%%%%%%%%%%%%%%%%%%%%%%%%%%%%%%%%%%%%%%%%%%%%%
\subsection{Hardware}
\label{subsec:hardware}
%The UPDRS scores of the PD volunteers were assessed by the same physician (our third author), just before data collection. We attached an iPhone on our volunteers’ hands using the same custom-made mounting glove (Fig. 1) from [15] and [16]. It consists of a perforated case into which the phone “locks”, and a wrist-supporting glove, both commercially available. The glove fits tightly on the volunteer’s hand and the case is tightly sewn on the glove using non-elastic thread, ensuring the stability of the device on top of the hand. With the device attached, each participant had to maintain each of two prescribed postures for 30 seconds, while acceleration and gyroscope data was recorded by the phone. The two postures we used were the same ones used during the clinical evaluation: a) “Extended”, i.e., seated with both hands extended in front of the torso (Postural Tremor of the Hands, component 3.15 of the MDS-UPDRS) and b) “Rest”, i.e., seated with both hands placed on the arms of the chair (Rest Hand Tremor, component 3.17 of the MDS-UPDRS). The procedure was then repeated for the subject’s other hand, in the same two postures. In the following, we will specify the combination of a patient’s hand (Right of Left upper extremity) during each position as rR, rL, eR, and eL for rest-right, rest-left, extended-right, and extended-left, respectively.
%The hardware setup was the same as the one used in our earlier work, [15] and [16]: 
%1.	An iPhone 4S with the latest iOS, with Internet access enabled, and screen orientation locked in vertical,
%2.	A web application to collect data from the smartphone’s sensors,
%3.	A web server to host the site and store the signals, and
%4.	A MATLAB application for processing the signals received at the server.
%As per the protocol described earlier, the volunteers were asked to maintain certain postures for 30 seconds, during which the application automatically collected the accelerometer and gyroscope data and sent them to our server. We then used the data to extract features which quantify and characterize the subjects’ tremor levels. The signals were sampled at 20Hz, which is sufficient to identify events occurring at 9Hz or less [20], such as PD-induced tremor. Our web application, being written in PHP and JavaScript, is entirely independent of the client’s hardware or software platform. It only demands basic prerequisites such as an embedded accelerometer and gyroscope and one of the most popular smartphone operating systems, iOS or Android. We successfully tested it on a Samsung Galaxy S4 and a Google Nexus 5, both running Android 4.4.2. 
%%%%%%%%%%%%%%%%%%%%%%%%%%%%%%%%%%%%%%%%%%%%%%%%%%%%%%%%%%%
\subsection{Software}
\label{subsec:software}
%%%%%%%%%%%%%%%%%%%%%%%%%%%%%%%%%%%%%%%%%%%%%%%%%%%%%%%%%%%
\subsection{Data Collection}
\label{subsec:dataCollection}
%%%%%%%%%%%%%%%%%%%%%%%%%%%%%%%%%%%%%%%%%%%%%%%%%%%%%%%%%%%
\subsection{Signal and Metrics}
\label{subsec:signalMetrics}
%For each volunteer’s session we obtained two signals from the phone’s sensors, the acceleration vector 〖α(i)=[α_x (i),α_y (i),α_z (i)]〗^T (in m/s2) and the  rotational velocity vector 〖ω(i)=[ω_x (i),ω_y (i),ω_z (i)]〗^T (in deg/s), with i denoting discrete time. The rotational velocity in practice should be more information-rich because it is constructed using both accelerometer and gyro data and is expected to capture more of the characteristics of the tremor. We applied a band-pass filter with cutoff frequencies of 1.5Hz and 9.5Hz, in order to exclude noise due to breathing, pulse, or any high-frequency sudden movements during the recordings. The spectral analysis of α(i) and ω(i) of a PD volunteer with typical Parkinsonian tremor is shown in figure 2. As expected, her acceleration and rotational velocity signal amplitude peaks at about 3-5Hz, which is consistent with the literature [20]. We used the acceleration and rotational velocity signals as in [16], to compute the following four metrics for each session:
%〖mag〗_α=∑_1^N▒‖α(i)‖^2  		(1)
%〖mag〗_ω=∑_1^N▒‖ω(i)‖^2  , 		(2)
%〖sd〗_α=∑_(i=1)^(N-1)▒∑_(κ∈{x,y,z})▒|α_κ (i)-α_κ (i+1)|  , 		(3)
%〖mAmp〗_ω=∑_(κ={x,y,z})▒max┬(4≦ ξ ≦7)⁡〖(ω_κ ) ̂(ξ)〗    , 		(4)
%where: 〖mag〗_α and 〖mag〗_ω are the sums of squared magnitudes of the acceleration, and the rotation rate vector respectively, and 〖sd〗_α, is the sum of absolute differences in the acceleration vector, summed over each of the three axes, x, y, and z. To compute the 〖mAmp〗_ω metric (4) we initially obtained the magnitude of the Fourier transform of each of the three axial components of the rotation vector ω(i), defined as (ω_κ ) ̂(ξ), κ∈{x,y,z}. We then determined each component’s maximum in the 4≦ ξ ≦7 Hz range (that range being consistent with the frequency of Parkinsonian tremor) and summed the three maxima [16]. 

%%%%%%%%%%%%%%%%%%%%%%%%%%%%%%%%%%%%%%%%%%%%%%%%%%%%%%%%%%%
%%%%%%%%%%%%%%%%%%%%%%%%%%%%%%%%%%%%%%%%%%%%%%%%%%%%%%%%%%%
\section{Signal Processing}
\label{sec:signalProcessing}
%%%%%%%%%%%%%%%%%%%%%%%%%%%%%%%%%%%%%%%%%%%%%%%%%%%%%%%%%%%
\subsection{Means Testing}
\label{subsec:meansTesting}
%Since our goal is to facilitate monitoring and diagnosis of PD-induced tremor, it is essential to establish that the metrics described in the previous section can be useful in differentiating the PD vs H populations. We used the non-parametric Mann-Whitney test to establish that the two populations have statistically different means in all four metrics on all four postures, rR, rL, eR, eL. As shown in Table III, all between-groups tests found significant differences between the mean scores of the metrics of the H volunteers compared to the mean of the PD volunteers. This suggests that the two populations (H and PD) have statistically different scores under every one of the signal metrics computed, and one may attempt to differentiate H vs PD subjects based on one or more of those metrics.
% There was no statistical difference in the subjects’ left vs. right mean scores within each group (see Table IX in Appendix). It is typical for PD patients to manifest the disease’s symptoms with some laterality, i.e., to a greater degree on one side, right or left. That is indeed the case with our PD volunteers because 19 of the 23 have differences between the sums of the UPDRS components concerning right vs left hand tremor indicating laterality of motor impairment. Although clinically observable, the Mann-Whitney test for the summed UPDRS scores for right vs left hands yields no statistical difference, with p=0.7327. In order to identify the laterality statistically, for each metric we summed the scores of both postures for right and left hand separately and calculated the absolute differences between them. As shown in the last four rows of Table III, for each metric, the absolute differences between hands for the PD group is statistically different from those of the H group. That means that the amount of difference between hands is not the same for PD and H, presumably due to the disease’s laterality.

%%%%%%%%%%%%%%%%%%%%%%%%%%%%%%%%%%%%%%%%%%%%%%%%%%%%%%%%%%%
\subsection{Features Identification}
\label{subsec:featuresIdentification}
%%%%%%%%%%%%%%%%%%%%%%%%%%%%%%%%%%%%%%%%%%%%%%%%%%%%%%%%%%%
\subsection{Correlation with Scale-based Metrics}
\label{subsec:correlation}
%In previous work [16] we attempted to establish the validity of our smartphone-based method of upper limb parkinsonian tremor quantification by running a Pearson product-moment correlation analysis between the UPDRS scores of the PD volunteers and their respective signal metrics. Table IV contains the results of the correlation analysis for the Rest posture scores. The numbers are slightly different from our previous study because the signals are now band pass filtered, as previously explained. In [16] we used the Rest posture data only, whereas here we were also interested in the Extension posture data for both hands. The correlation analysis of the Extension posture data (Table V) yields low coefficients (r<0.6) with low confidence (p>0.01). The results are better for the right hand but generally do not suggest good correlation between the UPDRS scores and the smartphone metrics. These findings show a connection between the manifestation of the action hand tremor and the hardware experimental setup. The fact that the resting tremor is identified consistently, whereas in the extended posture the measured tremor correlates weakly with the clinical assessment, is probably related to the effect of the mass of the smartphone on the dynamics of the hand/arm system.
%%%%%%%%%%%%%%%%%%%%%%%%%%%%%%%%%%%%%%%%%%%%%%%%%%%%%%%%%%%
%%%%%%%%%%%%%%%%%%%%%%%%%%%%%%%%%%%%%%%%%%%%%%%%%%%%%%%%%%%
\section{Machine Learning Model}
\label{sec:machineLearning}

%%%%%%%%%%%%%%%%%%%%%%%%%%%%%%%%%%%%%%%%%%%%%%%%%%%%%%%%%%%
%%%%%%%%%%%%%%%%%%%%%%%%%%%%%%%%%%%%%%%%%%%%%%%%%%%%%%%%%%%
\section{Discussion}
\label{sec:discussion}

%%%%%%%%%%%%%%%%%%%%%%%%%%%%%%%%%%%%%%%%%%%%%%%%%%%%%%%%%%%
\subsection{OFF - State Trials}
\label{subsec:offState}
%The 23 volunteers of the PD group who underwent the smartphone-based tremor measuring procedure were under medication, but the timespan from their last dose of L-DOPA was anywhere from 1 to 4 hours when they were tested. That means that some of them were at the peak of the drug’s effect while for others this was not the case. With an eye towards tracking the progression of PD, we wanted to see how our metrics would “react” to an alteration in a patient’s condition, such as that brought on by medication intake. We observed two PD volunteers, referred to as the PDDN group, who stayed in the clinic overnight and followed our experimental protocol both “off” and “on” medication (i.e., right before taking their medication in the morning and one hour after that). In Table VI we present the percent differences “on”-“off” in the four metrics, along with the UPDRS scores (“off” and “on”) for both PDDN subjects. We would expect those differences to be negative because we expect higher metric scores (more pronounced tremor) while off-medication or in a de novo state, and lower after the drug ingestion (“on”). From the UPDRS scores of the two volunteers it is clear that these two patients did not suffer from severe hand tremor. Their physician observed that the medication improved mostly the patients’ rigidity (which is not measured by our tool) and less so their tremor. However, it is encouraging to note that the readings of the smartphone’s sensors respond well and follow the expected negative trend of the changes in the UPDRS scores in the “on” state. The only discrepancy is observed in the eR position of volunteer B for all metrics, however for that position there were also no observed clinical changes in the UPDRS “on”-“off” as well.