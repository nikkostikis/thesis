\chapter{Automated Quantification}
\label{ch:autoQuant}
\pagestyle{fancy}
\fancyhf{}
\fancyhead[OC]{\leftmark}
\fancyhead[EC]{\rightmark}
\cfoot{\thepage}

%%%%%%%%%%%%%%%%%%%%%%%%%%%%%%%%%%%%%%%%%%%%%%%%%%%%%%%%%%%
%%%%%%%%%%%%%%%%%%%%%%%%%%%%%%%%%%%%%%%%%%%%%%%%%%%%%%%%%%%

\section{Complementing the PD diagnostic protocol}
\label{sec:complementing}
Diagnosis of \gls{PD} is a complex undertaking. The clinical evaluation mainstay is the face-to-face scale-based symptoms' assessment procedure performed by a medical expert in close collaboration with the patient and their caregiver. The most extensively tested and established method is the \gls{UPDRS} (Ebersbach et al, 2006), in its old format or the \gls{MDS} revision, with the \gls{HY} and \gls{SE} single item scales as adjunct tools. To avoid misdiagnosis, the evaluation process may be complemented with a differential diagnosis, where the physician will use criteria to exclude or support other conditions, such as secondary parkinsonism or other neurodegenerative diseases. When the clinical examination leaves little doubt that the patient suffers from \gls{PD}, the physician will prescribe dopaminergic medication, starting with dopamine agonists or metabolism blockers (\gls{MAO-B}), before administering levodopa. The patient's response to the medication over a period of time will eventually confirm or challenge the \gls{PD} diagnosis. If the response is not good and the symptoms remain, the physician will likely have to re-evaluate and repeat the clinical assessment. Should the medical expert find it necessary, they may request for neuroimaging examination, usually a \gls{PET} or \gls{SPECT} scan, to acquire additional information on the patient's neuronal activity.

The steps described above have been and will probably remain the main \gls{PD} diagnosis pipeline in the clinical practice for many years to come. That is mostly because the cause of idiopathic \gls{PD} is not known and the cardinal symptoms, broadly referred to as parkinsonism, are shared across many other pathological conditions, which require completely different treatment approaches. The scientific community has tried to fill some of the gaps in the diagnostic path or expedite parts of it, in order to alleviate patients' symptoms sooner, improve their daily life and, as an added bonus, reduce the cost of treatment involved, which can be quite substantial.

%%%%%%%%%%%%%%%%%%%%%%%%%%%%%%%%%%%%%%%%%%%%%%%%%%%%%%%%%%%
%%%%%%%%%%%%%%%%%%%%%%%%%%%%%%%%%%%%%%%%%%%%%%%%%%%%%%%%%%%

\section{Axes of Innovation}
\label{sec:axes}
Regarding PD, where the diagnostic protocol is contingent on human expertise and experience, scientists are trying to find tools that could objectively measure the disease's manifestations and create a distinctive symptom profiles that will help identify it among other similar conditions. Although \gls{PD} has a plethora of symptoms and affects various aspects of the patients' daily life, it is mainly the disease's motor symptoms which provide a promising platform for objective quantification and profiling. Non-motor symptoms, though numerous and debilitating, cannot be easily measured and recording them relies heavily on the patients' testimony (see Figure ~\ref{fig:UPDRS1A} where we describe the algorithm to score part 1A of \gls{UPDRS}), which is rather subjective. 

There are multiple reasons supporting the need to quantify \gls{PD} motor symptoms and we identify the following six axes driving the innovation:

\begin{enumerate}
\item \textit{Cost efficiency}. One major concern regarding healthcare is the increasing cost associated with diagnosis and treatment of chronic diseases. Solutions reducing that cost even by a small percentage can both alleviate the patients' financial burden and contribute to proper re-allocation of resources by national and private insurance providers to other problematic areas. \gls{PD} management, being a chronic endeavor, needs cost efficient solutions that healthcare and insurance providers could adopt.
\item \textit{Availability}. People living in rural areas or under-developed and developing countries have trouble accessing medical experts and facilities. Scientists work towards providing tools that can make services and equipment more easily accessible to those people. Tele-medicine and remote diagnostic tools are being developed towards that end. \gls{PD}, being a disease affecting almost 2\% of individuals over 55 years of age worldwide (see paragraph \ref{subsec:demographics}), requires tools that will be easily available in both urban and rural areas. Exotic equipment is seldom available in remote locations.
\item \textit{Ease of use}. Medical equipment is generally difficult to use and most of the times requires trained personnel to operate it. Solutions that are functional and easy to use can be incorporated into clinical practice significantly faster and with little training. Methods requiring a lot of training and expertise would probably be limited to experimental application. 
\item \textit{Quality of diagnosis}. This axis includes early diagnosis, accuracy in the decision making, and effective differentiation between similar conditions. In particular, as discussed earlier, parkinsonism is a common complex of symptoms for many diseases (see paragraph \ref{sec:misdiagnosis}). A high quality \gls{PD} diagnosis would entail precise identification of the symptoms and their characteristics, little to no time wasted on examining the response to the medication, and proper exclusion of other conditions. 
\item \textit{Effective monitoring}. The stages that follow the initial diagnosis are usually equally if not more important to the management of the disease's progression. A tool or method that would let the physician check frequently patients' progress, would create unimaginable possibilities to assess micro-variations, adjustments and combinations of the available medications. \gls{PD} patients in particular, could benefit greatly from a monitoring system that would collect data on their motor-related symptoms, and communicate them to their physician. For patients in the later stages of the disease, who have been on dopaminergic medication for years and have started experiencing the drug-related dyskinesias (\gls{LiD}), the untimely wearing-off and the irregular fluctuations, detailed information on how they are experiencing these phenomena daily could help their physicians design improved and adjusted personalized medication, nutrition and activity regimens to mitigate their discomfort. 
\item \textit{Biobanking}. Effective symptom monitoring, data on daily activities, medication performance and disease progress are recorded and, appropriately anonymized, remain available to medical researchers, pharmacists and biologists. These data stores can help scientists plan more effective treatment regimens, design detailed diagnostic protocols and deliver personalized and sophisticated pharmaceuticals. Using automated assessment tools, the adjustments and improvements made to the approaches of \gls{PD} symptoms alleviation for particular patients in combination with their profiles and general motor characteristics could be stored in centralized and easily accessible anonymized data pools for future cases to be treated efficiently even faster and in a more personalized and consistent manner.
\end{enumerate}

\hlorange{Table here summarizing the axes of innovation}

%%%%%%%%%%%%%%%%%%%%%%%%%%%%%%%%%%%%%%%%%%%%%%%%%%%%%%%%%%%
%%%%%%%%%%%%%%%%%%%%%%%%%%%%%%%%%%%%%%%%%%%%%%%%%%%%%%%%%%%

\section{Quantification via Accelerometry}
\label{sec:accelerometry}
During the past few years there have been several efforts to establish computer and sensor-assisted methods for evaluating abnormalities in human motor activity and functionality. As stated in (Patel et al, 2012), inertial sensors can be used to keep patients connected to their physicians, in a non invasive, easy to use, ubiquitous manner. Advancements in microelectronics have miniaturized electronic circuits and power modules, bringing sensor-packed platforms closer to the body via wearable pieces of apparatus. Flexible \gls{ECG} and \gls{EMG} sensors, gyroscopes, accelerometers, fabric-embedded leads and ambient sensors, can provide vast amounts of data on a daily basis, and create huge potential for research on how a patient's condition fluctuates. In the context of motor patterns quantification, accelerometers and gyroscopes are particularly promising. 

One of the first reports on accelerometers' usefulness as clinical assessment tools was (Auvinet et al, 1999). In that study two accelerometers held on the volunteers' waist by a semi-elastic belt were used to evaluate the reproducibility, sensitivity, and specificity of gait analysis. The authors concluded that the devices' performance was satisfactory enough to be used by clinicians. The same year, a study in ambulatory monitoring (Foerster et al, 1999), validated the use of accelerometry for the detection of posture and motion. 

Accelerometers have since been heavily used in experimental medical applications for tremor quantification (Patel et al, 2009; LeMoyne et al, 2013; Kostikis et al, 2015), fall detection (Lim et al, 2014; Li et al, 2009), gait analysis (Klucken et al, 2013), motor state assessment (Keijsers et al, 2006), and heart-rate variability identification (Phan et al, 2008). They have also been used to identify and remove noise from signals collected via other sensing methods such as photoplethysmography (Mullan et al, 2015). Along with the actual functional advantages of accelerometers and the insight they offer into human motion, accelerometers are cheap, easy to use, and can be easily embedded in larger platforms and communicate with other modules.

For \gls{PD} in particular, accelerometers quantifying tremor, posture and gait are used in the majority of proposed novel diagnostic and monitoring applications.

In (Manson et al, 2000) the authors built a portable device based on three uniaxial accelerometers, an amplifier, a battery pack and a data recorder. The accelerometers were mounted on the shoulders of 26 volunteers (16 \gls{PD} patients and 10 healthy control subjects), and the rest of the equipment was strapped to their waist. For the \gls{PD} patients, the accelerometers were mounted on the shoulder of the most affected by the \gls{PD} symptoms side (see paragraph \ref{subsec:laterality}), whereas for the control subjects the shoulder of the dominant side was preferred. All the subjects performed various tasks, such as talking, walking and writing while wearing the sensors. \hl{Correlation analysis showed that accelerometry could be used to successfully assess levodopa induced dyskinesias}. 

The authors of (Thielgen et al, 2004) conducted a longitudinal study on \gls{PD} tremor using accelerometers on 30 patients. They recorded rest and postural tremor under standardized settings and also performed full 24-hr recordings. They repeated the recordings for 21 of the patients after less than 3 weeks, during which time the patients received personalized tailored drug treatment supplemented with specific physiotherapy, ergotherapy measures, and psychotherapeutic counseling. The differences between the two recordings were noticeably different, with the 24-hr monitoring of after 3 weeks being significantly different than those prior to the personalized treatment. \hl{This research proved that accelerometry could be used to detect the effects of treatment regimens and their adjustments over time}.

Drug-induced dyskinesia was assessed in (Patel et al, 2006) using a network of eight accelerometers placed on the upper and lower limbs. \hl{By means of clustering analysis the authors were able to identify changes in the severity of dyskinesia during patients' motor fluctuation (ON-OFF) cycles}. 

Based on the findings of (Foerster et al, 1999), the authors of (Salarian et al, 2004) introduced a new platform consisting of body-worn accelerometers and gyroscopes for gait analysis and assessment in \gls{PD} patients through ambulatory long-term monitoring. The authors recruited 10 \gls{PD} patients who had undergone Deep Brain Stimulation of the \gls{STN} via stereotactic surgery \hl{(see Appendix }\ref{app:appTreatment}\hl{ for details)} and 10 age-matched healthy control subjects. The volunteers were asked to walk multiple times on a 20m walkway. Patients' recordings were obtained both with \gls{DBS} on and off. \hl{PD patients had significantly different gait parameters compared to the control subjects. The effect of the DBS was identified and most of the variables extracted from the gait analysis correlated well with the patients' UPDRS equivalent scores}. The same wearable sensor network was used in (Salarian et al, 2007) to quantify tremor and bradykinesia in \gls{PD} patients. In that study 10 \gls{PD} patients followed a 45 min protocol of 17 typical daily activities, while wearing the sensors. Their motion characteristics were compared to those of 10 age-matched healthy controls. In a second study, 11 \gls{PD} patients were monitored with the sensors while moving freely for 3 to 5 hours. \hl{The results from both studies showed that the sensors were both sensitive and specific to parkinsonian symptoms related to tremor and bradykinesia}. The same hardware setup was once again used in (Salarian et al, 2010), this time for the successful instrumentation of the \gls{TUG} test. \hl{In that study, although a timer did not reveal any significant difference in the TUG test performance of 12 early PD patients and 12 healthy control subjects, the accelerometers mounted on their limbs and waist, helped identify significant difference in cadence between the two groups}. The experimental hardware setup used in all the aforementioned studies in this paragraph was later commercialized under the brand GaitUp.

Extensive work was done in (Patel et al, 2006; 2007; 2009) on analyzing the feature space for monitoring \gls{PD} patients' symptoms and motor fluctuation. In those studies the authors used a wearable network of 8 sensors, one on each part of each limb. The sensors were Intel's \gls{SHIMMER}, which will be discussed a later paragraph. 

The authors of (Weiss et al, 2011) used a body-worn accelerometer, placed at the lower back of 22 \gls{PD} patients and proved that the data thus collected could \hl{estimate the stride-to-stride variability and help assess the quality and consistency of walking in PD patients}. Two years later it was proven that \hl{long-term recordings from body-worn accelerometers could be used to evaluate fall risk (Weiss et al, 2013)}. The volunteers wore the accelerometers for a period of three days and based on their fall history, were correctly identified as non-fallers and fallers\footnote{Non-fallers suffer from less than 2 falls in a year and fallers suffer from more than 2 falls in a year.}. 

An important piece of the puzzle of using body-worn accelerometers to assess \gls{PD} symptoms, sometimes even for whole days is the compliance of the patients themselves. In (Fisher et al, 2016) the researchers found that patients' concordance with wearing the sensors can be relied upon, even for long-term monitoring. As we will discuss later, this agrees with our experience during our own clinical trials.

\textcolor{BurntOrange}{Table X} summarizes the most notable research results highlighting the value of accelerometers in \gls{PD} symptoms quantification. 


%%%%%%%%%%%%%%%%%%%%%%%%%%%%%%%%%%%%%%%%%%%%%%%%%%%%%%%%%%%

\subsection{Professional Accelerometry Sensor Networks and Platforms}
\label{subsec:sensorNetworks}
In the previous paragraph the work of researchers who built their own custom experimental platforms with accelerometers to prove the concept that these sensors can be used successfully in medical practice was discussed (Manson et al, 2000; Auvinet et al, 1999; Salarian et al, 2010). The specifications and sophistication of these platforms vary. Some are limited to one or two sensors connected via cables, whereas in some cases experimental platforms of professional grade and modular architecture have been developed and later commercialized.

In this paragraph we will review the products of companies who have built feature-rich, energy-efficient, wearable platforms with embedded inertial sensors capable of monitoring acceleration, position and rotation. These platforms can be sold as products to clinical researchers. They are standardized, modular and ready to be used in a clinical setting to objectively assess human gait and quantify motion characteristics. Postural and motor traits which until recently could only be estimated empirically by expert physicians can now be evaluated and explored with specifically designed wearable sensors. 

An example of such commercialized, wearable sensor systems is Intel Digital Health Group's product called Sensing Health with Intelligence, Modularity, Mobility, and Experimental Reusability platform, known as \gls{SHIMMER}\footnote{http://www.shimmersensing.com}. The \gls{SHIMMER} basic module is a 65x32x12mm device incorporating an accelerometer, a gyroscope, a magnetometer and a pressure sensor. The device can be adapted to embed a 5-lead digital electrocardiogram (\gls{ECG}) or photoplethysmography (\gls{PPG}), heart-rate (\gls{PPG-HR}), electrodermal activity (\gls{EDA}) and galvanic skin response (\gls{GSR}) sensors. \gls{SHIMMER}'s accelerometry capabilities have been used to successfully identify \gls{PD} characteristics (Patel et al, 2007; Lorincz et al, 2009; Barth et al, 2011). Besides monitoring \gls{PD} samples \gls{SHIMMER} could be used in other scientific or professional context to measure physical activities and phenomena, such as heart rate variability and athletic performance. The basic module can store up to 32GB of data and can communicate wirelessly via Bluetooth with other devices. 

Another example of a standardized, wearable, FDA-cleared, sensor platform, is Kinesia One and its variant Kinesia 360\footnote{http;//www.glneurotech.com/kinesia/}. Both products include wearable accelerometry sensors worn on upper and lower limbs to assess tremor, bradykinesia and dyskinesia. The Kinesia One package includes a wearable sensor and a tablet preloaded with a special application which illustrates guidelines on tasks the patient must perform while wearing the sensor, to have his symptoms assessed. The sensor is worn on the patient's finger and it wirelessly transmits data to the tablet's application. The application then contacts a centralized secure server where it uploads the patient's anonymized data where clinicians and researchers can have access. The patient's privacy is protected. The Kinesia 360 package is used for long-term monitoring. It includes a tablet preloaded with a special application and two sensor bands to be worn on the wrist and ankle. The bands transmit data to the application wirelessly and the patient voluntarily records disease-specific diary information, such as sleep time, medication and symptoms assessment, so that a complete profile can be created. The patient typically wears the bands during the day, from the time he wakes up, until just before he goes to bed at night. Upon completion of the recording the data are transmitted securely to a server, where clinicians and researchers can access it. The patient's privacy is once again protected. The Kinesia products have been used in studies to successfully assess \gls{PD} motor symptoms (Giuffrida et al, 2009).

In the previous section we mentioned the work of Salarian and his colleagues. The hardware used in (Salarian et al, 2007) and (Salarian et al, 2010) was standardized and commercialized under the brand GaitUp\footnote{http://www.gaitup.com}. The sensors are called Physilog, are of similar dimensions to \gls{SHIMMER} sensors and feature accelerometers, gyroscopes, magnetometers and barometric pressure sensors. Bluetooth streaming is also an option offered by the higher-priced modules. Mariani et al (2013) used Physilog modules to assess the gait and turning \gls{TUG} test of \gls{PD} patients and found interesting features for identifying patients' symptoms, ON and OFF states. 

A portable gait and balance analysis platform named Mobility Lab is manufactured by Ambulatory Parkinson's Disease Monitoring (\gls{APDM})\footnote{http://www.apdm.com}. It is a modular product that can consist of multiple sensor modules called Opals. These modules can be worn alone or organized into networks, to measure the motion characteristics of a subject's full body. Opals embedded in the Mobility Lab platform have been used in various studies to identify freezing of gait during the \gls{TUG} test (Mancini et al , 2012), quantify leg dyskinesias (Ramsperger et al, 2016) and evaluate the effects of levodopa and its side-effects in balance and gait (Curtze et al, 2015; Baston et al, 2016; Elshehabi et al, 2016). The platform's validity as a monitoring technology for \gls{PD} was established by Godinho et al (2016), and Mancini and Horak (2016). 

One of the most promising efforts on commercializing accelerometry for \gls{PD} symptoms tracking is the Kinetigraph, by Global Kinetics Corporation\footnote{http://www.globalkineticscorporation.com.au}. The Kinetigraph is built to be used like a Holter monitor. Much like the ambulatory electrocardiography device  developed by Norman J. Holter (Ioannou et al, 2014), the single collector resembles a watch and must be worn by the patient upon his physician's suggestion for seven days. After that period of time the device must be returned to the physician who will upload the signals recorded and receive a detailed report on how his patient performed. The device is \gls{FDA} approved and \gls{CE} marked, and according to the company's website  is being used in many clinical trials. 

In (Lorincz et al, 2009), the authors, based on their previous work (Patel et al, 2007; 2009), standardized the hardware setup they had previously used and introduced an experimental platform called Mercury. Mercury is described as a wearable wireless sensor platform used for motion analysis of patients with parkinsonism (primary and secondary) and other movement disorders. Consisting of 8 \gls{SHIMMER} sensors, the Mercury architecture also includes a base station for data processing, an application driver and specifically developed signal collecting and processing software. The creators of Mercury claim that any sensor could be used instead of the \gls{SHIMMER} sensors, however this claim is not validated. Based on the same \gls{SHIMMER} modules, another team of researchers introduced the eGait platform (Klucken et al, 2013). EGait is a platform for automatic analysis of gait using sensors and cameras. Its creators claim it can be used for diagnosis, clinical study and treatment monitoring\footnote{http://www.egait.de}.

A custom-made, high-grade, experimental, accelerometry platform was developed under the \gls{PERFORM} project (Cancela et al, 2013; Tzallas et al, 2014; Cancela et al, 2014). The multi-parametric system for the continuous effective assessment and monitoring of motor status in Parkinson's disease and other neurodegenerative diseases, as they explicitly call it, is a platform composed of five ANCO 3-DoF accelerometers, one placed at each limb (i.e. wrists and ankles) and one at the patient's waist. A central storage unit is also strapped on the patient's waist and controls all the nodes. The central unit can also alert the patient for medication intake, and send \gls{SMS} messages, should an emergency button be pressed or if a fall is detected. The nodes transmit the data to the central storage unit wirelessly through a \gls{UDP}-like protocol, with no re-transmission capabilities. Lost data packets are simply acknowledged by means of the timestamps. The system is used similarly to the Kinesia 360, with the data being transferred via \gls{USB} to a PC application at the end of the recording session, which would typically last a day. The processing of the data is done locally on the PC and sent over the Internet to the centralized hospital unit, where the clinician can review the results and assess the patient's progress. \gls{PERFORM} has been used to successfully classify \gls{PD} patients through long-time monitoring (Cancela et al, 2014).

In \textcolor{BurntOrange}{Table X} we summarize the most widely used high grade body-worn sensor platforms and their characteristics.

%%%%%%%%%%%%%%%%%%%%%%%%%%%%%%%%%%%%%%%%%%%%%%%%%%%%%%%%%%%

\subsection{Smartphone-Based Accelerometry}
\label{subsec:smartphones}
The approaches discussed in the previous section pertain to platforms designed and built specifically for use in clinical settings and trials. Most of the aforementioned platforms comprising special-purpose hardware, deliver robust results (Godinho et al, 2016). \gls{SHIMMER}, Kinesia, Mobility Lab and Kinetigraph are products with great potential to benefit \gls{PD} patients, as more research is done and the protocols of clinical adoption are refined. However, these devices are relatively exotic, inaccessible to the general public and quite expensive. The Kinesia One and 360 are available only through specific clinics, one \gls{SHIMMER} sensor with a dock and the software costs \EUR{500}, a Physilog sensor with its software can be rented for a month for about \EUR{230} and bought for \EUR{1400}. It should be noted that most of the successful research done using these products involved body-worn networks of multiple sensors, which means that the cost to achieve the reported results can be quite high. Even though they do provide researchers with valuable information, these devices have high complexity in terms of hardware and software, high cost, and so far lack standardized, approved and widely accepted protocols.

The potential of accelerometers for assessing kinematic features has been established, and inevitably researchers started to explore more ubiquitous and low-cost accelerometry solutions. Smartphones are a tantalizing platform for such an undertaking, placing the required instrumentation at the hands of an ever-growing number of people worldwide. Nowadays these devices are essentially omnipresent, with \textcolor{BurntOrange}{4 billion devices in circulation} in 2016 and an estimation of \textcolor{BurntOrange}{6 billion units owned globally by 2020}\footnote{http://www.cnbc.com/2017/01/17/6-billion-smartphones-will-be-in-circulation-in-2020-ihs-report.html - link accessed on 03/28/2017}. Smartphones feature multiple sensors and connectivity options, and can be used to collect spatiotemporal signals, such as acceleration, rotation and position. This creates intriguing possibilities for smartphone-based applications which will quantify movement disorders' symptoms, such as the \gls{TRAP} complex, as it was defined in section \ref{sec:trap}.

The work of \hl{(LeMoyne et al, 2010)} pioneered in proving the viability of a smartphone-based accelerometry solution to record tremor and transmit the signal wirelessly to a remote station for post-processing. An iPhone 3G was mounted to the hand's dorsum of two volunteers, a healthy individual and a \gls{PD} patient. An application installed on the device recorded the acceleration produced by the hand's action postural tremor for multiple trials of 10sec. Each trial's signal was sent via email to a remote host for post-processing. Spectral analysis of the \gls{PD} patient's acceleration signal revealed predominant frequencies around 5.3Hz, 7.7Hz and 10.4Hz. Equivalent analysis of the signal recorded from the healthy volunteer showed no predominant frequencies. \hl{This study showed that a smartphone-based accelerometry tool could be used to identify the characteristics of the PD induced postural tremor}. 

LeMoyne's work accentuated another important feature of smartphones when used in telemedicine. The signal recorded from the volunteers was stored locally and after the trials were completed, it was sent via email to a remote host, which was many miles away from where the recordings were conducted. LeMoyne did that to expose the disadvantage of the specialized platforms discussed earlier, which only support a few meters of wireless connectivity, with the need to download or stream the data from the sensor to a base station located nearby. Smartphones offer virtually limitless connectivity, allowing the signal to be transmitted, either synchronously or post-collection to anywhere in the world, removing any spatial constraints. 

In 2011 we took ubiquity one step further in a prototype tool that could collect acceleration and rotation signals from a smartphone through JavaScript, without the need for installing a custom application or taking any extra steps to upload the data to a remote server \hl{(Kostikis et al, 2011)}. The tool we built was verified by (Kostikis et al, 2014), where we ran correlation tests between signals collected with a smartphone and the \gls{UPDRS} scores of 23 \gls{PD} patients. Our results regarding quantification of tremor and identification of pathological motor patterns were very promising and will be discussed in detail in chapter \ref{ch:smartphone}. 

Another important milestone in the research regarding smartphones and tremor quantification was reached by \hl{(Daneault et al, 2013)} which proved that a smartphone could be used as a standalone platform for tremor detection and monitoring. The authors of that work used an arm-mounted BlackBerry smartphone to measure, collect and process the acceleration signal through its embedded sensors and central processing unit. The characteristics derived from the acceleration signal were both time- and frequency-domain related. The results from the processing performed on-board the device were compared with both the results from post-processing the smartphone's raw signal on a personal computer, and the post-processing of the signal collected with a one-axis accelerometer taped on the smartphone and recording simultaneously. Each participant performed four tasks with the smartphone and separate taped accelerometer mounted on their hand:

\begin{enumerate}
\item sitting with the arm hanging by the side, assessing rest tremor,
\item sitting with the arm extended in front, parallel to the ground, assessing postural tremor,
\item sitting and keeping arms in front of chest, bent on the el, with palms facing down, while trying to keep the tips of the fingers touching, assessing action-intention tremor, and 
\item starting in the same position as in task 2 and then bringing the phone-mounted arm to the ear and back, assessing kinetic tremor. 
\end{enumerate}

\hl{Correlation analysis showed that the smartphone on-board processing yielded similar results when compared to the post-processing of the phone's raw signal with all tasks' correlation coefficients being above 0.90 for all calculated characteristics}. Similarly, the smartphone's on-board processing results correlated well with the post-processing results of the separate accelerometer's signal, with correlation coefficients ranging from 0.88 to 0.95 for all tasks and characteristics, except for the results from task 4. However the low correlation of the kinetic task based metrics was deemed insignificant because the goal was to mainly assess abnormal postural and rest tremor. The most valuable features extracted from the acceleration signals were the time domain related ones. In order to further validate their method, the authors recruited 16 patients, 12 diagnosed \gls{PD} patients, 3 with \gls{ET} and 1 suffering from Multiple Sclerosis (\gls{MS}). They performed a correlation analysis between the results from the smartphone acceleration signal collection and on-board processing, and a custom scale they created to assess the patients clinically. They found a strong relationship between the amplitude of tremor measured by the device and the one measured with the custom clinical scale. The smartphone calculated consistently different amplitude mean values corresponding to each level of the scale, indicating that it could in fact be a valid method of creating tremor profiles over time. 

In \hl{(Araújo et al, 2016)} it was confirmed that acceleration \hl{signal collected from a smartphone, more specifically an iPhone, can be used as a reliable alternative to the EMG for tremor frequency assessment}. They tested 22 patients diagnosed with \gls{PD}, \gls{ET} and Holmes' tremor, using three data collection methods; an iPhone, a separate accelerometer and an \gls{EMG} needle. The results were indicative of the potential of smartphone embedded accelerometers, with a correlation coefficient as high as 0.8 among the three methods. 

In 2015 our paper on a smartphone-based upper limbs tremor quantification tool was published \hl{(Kostikis et al, 2015)}. In it we extended the work done previously (Kostikis et al, 2011; Kostikis et al, 2014) to attest the potential and efficacy of smartphone sensing capabilities in clinical settings. Conducting a clinical trial consisting of 25 \gls{PD} patients and 20 age-matched healthy individuals, we used our tool to build, train and validate a machine learning classification model. Our model scored 82\% sensitivity and 90\% specificity. The methods and results from this work will be discussed in detail in chapter \ref{ch:smartphone}. 

The latest effort in quantifying \gls{PD} symptoms utilizing a smartphone is cloudUPDRS\footnote{http://www.updrs.net/}. It is a mobile application under development since 2012, by a group of academic and industrial partners, Benchmark Performance Ltd, Retechnica Ltd, Audience Focus Ltd, the University College London and Birkbeck university of London, among others. Proof of concept was established in (Kassavetis et al, 2015), where, as the authors claim, for the first time a mobile application was used to holistically assess \gls{PD} motor symptoms. A clinical trial is currently underway to prove the solution's validity and usability\footnote{https://clinicaltrials.gov/ct2/show/NCT02937324}. 

Testing via smartphones is generally advertised as a ubiquitous solution, taking little time to complete, while requiring very little expertise and ideally no medical supervision to conduct. 
In (Stamate et al, 2017) the researchers behind cloudUPDRS used deep learning to make sure their solution can be used unattended to produce accurate tests in the least amount of time. Mre specifically, they incorporated deep learning algorithms to accomplish two goals: 

\begin{itemize}
\item to promote the users' adherence to the assessment protocol in order to ensure high quality data collection, and
\item to reduce the assessment procedure's duration by personalizing the whole process according to each patient's symptoms profile. 
\end{itemize}

In \textcolor{BurntOrange}{Table X} we present the most notable milestones in using smartphones accelerometry to quantify \gls{PD} tremor. 

%%%%%%%%%%%%%%%%%%%%%%%%%%%%%%%%%%%%%%%%%%%%%%%%%%%%%%%%%%%
%%%%%%%%%%%%%%%%%%%%%%%%%%%%%%%%%%%%%%%%%%%%%%%%%%%%%%%%%%%

\section{Quantification via Handwriting}
\label{sec:handwriting}
A completely different approach to \gls{PD} symptoms' quantification is the assessment of handwriting tasks. Such approaches were initially incentivized by the fact that micrographia is among the symptoms of PD. Micrographia was defined by Wilson (1925) as ``an obvious reduction in size of the lettering of the writer in comparison with his calligraphy before the development of the organic lesion effecting the change.'' Wilson distinguishes between two types, ``consistent'' micrographia, where there is a global reduction in lettering size and ``progressive,'' where the writer cannot maintain normal sized letters for more than a few characters. What is interesting about micrographia is that is has no connection to other \gls{PD} symptoms. It appears to be uncorrelated to tremor and rigidity and is only indefinitely related to akinesia. In (McLennan et al, 1972), 10 - 15\% of 800 \gls{PD} patients examined were found to suffer from micrographia. Another study identified the symptom in 29\% of the \gls{PD} patients they examined (Contreras-Vidal et al, 1995). What is important though is that the symptom could be identified through human observation up to four years before the occurrence of measurable tremor or rigidity. 
Both types of micrographia, consistent and progressive are positively affected if the patient concentrates his attention on the writing task. Only consistent micrographia can be ameliorated by the use of levodopa, whereas the progressive type is resistant to medication. This is a strong indicator that lesions in different structures of the brain cause the two types. So far, the pathophysiology of parkinsonian micrographia is not completely understood. Automating and systematizing its quantification could promote early diagnosis and offer insight into the pathophysiology behind it.

Many researchers have experimentally confirmed that letter sizing is a problem for \gls{PD} patients. In (Teulings et al, 2002) the authors used a display digitizer to examine \gls{PD} patients' response to visual distortion of their handwriting. The researchers instructed the participants to perform several handwriting tasks providing them with visual sizing goals, but they purposefully distorted the vertical axis of their traces. The participants' response to the deceptive distorted visual feedback of their performance would show how well, if at all, patients could adapt to the feedback and adjust their handwriting pattern. The normal response to the visual distortions, in order to abide by the size goals, would be to increase the stroke size when the trace was reduced and decrease the stroke size when the trace was enlarged. The control participants responded normally. Remarkably enough, the \gls{PD} patients had the inverse response. They performed larger strokes when their traces were enlarged and smaller strokes when their traces where reduced. This shows that they had difficulty in planning the necessary stroke to accomplish the size goal and instead followed the ongoing distorted visual feedback, ultimately failing the size goals. 

Interesting conclusions on \gls{PD} micrographia were also drawn by the authors in (Van Gemmert et al, 1999; 2001; 2003). These studies showed that, given an initial target pattern, when the number of words to be written increased, \gls{PD} patients reduced their stroke size and would start having problems modulating letter sizing upwards from 1.5cm, whereas healthy individuals are able to modulate stroke size without changes in duration and speed for sizes up to 2cm. In (Van Gemmert et al, 2003) the authors recruited 13 \gls{PD} patients and an equal number of age-matched healthy volunteers and instructed them to write on a digitizer tablet with a screen behind it a cursive ``11111111'' pattern, or a cursive ``$\ell$ $\ell$ $\ell$ $\ell$ $\ell$ $\ell$ $\ell$ $\ell$'' pattern in five different sizes, 1, 1.5, 2, 3, and 5cm. The desired sizes for each trial were briefly displayed on the screen and then hidden the moment the pen touched the digitizer. The digitizer itself was also covered so that the participants could not monitor their hand while drawing. The participants were requested to perform the writing tasks at a comfortable speed and then again as fast as possible. Each participant performed 12 trials. The analysis of the drawn patterns showed that \gls{PD} patients had lower comfortable speed than controls, slightly faster ``as fast as possible'' speed, and significantly smaller stroke sizes, regardless of the speed. \gls{PD} patients' microgaphia was more evident for size conditions above 2cm. When writing at comfortable speed, \gls{PD} patients managed to achieve the sizing goals much better than when writing as fast as possible. That is an indicator that concentration can mitigate micrographia, whereas automated and less controlled motion can suffer. 

However, handwriting tasks do not only involve writing text and are not only used to evaluate the size of the letters a \gls{PD} patient writes. The term ``PD dysgraphia'' was introduced by the authors of (Letanneux et al, 2014) to accommodate any abnormal behavior in the mechanics of handwriting skills. They concur that all motor impairments of \gls{PD}, such as the \gls{TRAP} complex, reduced visuospatial perception, and motor coordination deficiencies can actually contribute to pathological handwriting kinematics, in combination with and beyond writing size. This new definition implies that the same impairments resulting in micrographia, may be present and detectable through handwriting tasks, even when stroke sizing is not observably affected. Dysgraphia assessment would entail various and more sophisticated features being extracted from even simple drawing movements, and signal analysis going beyond stroke size and duration to identify discriminating characteristics and reveal early and subtle impairments. 

In (Luciano et al, 2016) the authors revisited the spiral test and recruited 138 \gls{PD} patients (50 of which were early-diagnosed, i.e., less than 5 years), and 150 healthy volunteers to draw spirals on a paper placed on a digitizer recording the two dimensional displacement and the pressure. They specifically ran the test for the \gls{PD} patients during their ON phase, when the medication was successfully ameliorating the symptoms, to explore the potential of the test to yield discrimination performance even for milder forms of \gls{PD}. The authors also tested both volunteers' hands, something not common in handwriting task-based tests. By doing that, they were able to exploit \gls{PD}'s asymmetric symptom manifestation and improve the test's performance. 

In (Flash et al, 1992) the authors outlined a protocol to analyze the kinematic characteristics of simple distally and proximally directed movements using a stylus and a digitizer, with \gls{LED} illuminated targets. They performed their experiments both allowing for visual feedback, where the participants could see the trajectory drawn and monitor their hand, and without visual feedback, where the participants could only see the \gls{LED}-marked targets. The participants were instructed to move the stylus given a particular trigger, that is the LED would flash once and then flash again and stay on for the duration of the movement. The participants would have to react to the second flashing as soon as that took place. The first flash would allow them to plan before execution. The condition of presence or absence of visual feedback, the simplicity of the task and the goal oriented approach allowed the researchers to go beyond the typical handwriting assessment, where the main characteristic would be sizing of the glyphs, and explore features concerning the trajectory, reaction time, final error, duration, acceleration and deceleration, and their correlations. Among their findings were the following interesting points:

\begin{enumerate}
\item Only for the distally directed movements and only \gls{PD} patients had substantially larger final target errors without visual feedback than when they had feedback of their hand and its trajectory. The error between efforts without and with feedback was not so different with regards to the proximally directed movements. \hlpink{That means that the mechanics of the two motion patterns, flexion and extension, are not the same, extension in this study being more vulnerable to the absence of visual feedback}. 
\item \hlpink{The trajectory velocities for PD patients were lacking smoothness, suffering from many ``corrections.'' It is particularly worth mentioning that these oscillatory effects were non-rhythmic and therefore not directly associated to tremor}.
\item The reaction time for \gls{PD} patients was twice as much as the time of the control subjects. Once again the target location was important; the average reaction time for distally directed movements was significantly longer (433ms longer) than the respective time in proximal efforts, indicating once again the difference in the two motion patterns' mechanics. Extension appeared to be more susceptible to impairment.
\end{enumerate} 

In 2017 our paper on a machine learning-based tool for the identification of simple line-drawing tasks performed by \gls{PD} patients was published (Kotsavasiloglou et al, 2017). In it, we described an easy to follow protocol for the assessment of the performance of \gls{PD} patients when drawing simple lines on a Wacom Bamboo pen tablet\footnote{Wacom pen tablets, accessed 05/10/2017 at:\url{http://www.wacom.com/en-us/products/pen-tablets}}. As will be discussed in detail in chapter \ref{ch:handwriting}, our analysis showed that even a simple line-drawing task, once digitized, can generate valuable metrics for the detection of \gls{PD} symptoms, and over 80\% accurate classification of pathological and healthy trajectories. 

%%%%%%%%%%%%%%%%%%%%%%%%%%%%%%%%%%%%%%%%%%%%%%%%%%%%%%%%%%%
%%%%%%%%%%%%%%%%%%%%%%%%%%%%%%%%%%%%%%%%%%%%%%%%%%%%%%%%%%%
\section{Our Approach}
\label{sec:ourApproach}
As was already mentioned, in a similar vein with some of the aforementioned studies we conducted our research on the quantification of \gls{PD} symptoms using \gls{IMU} sensors and handwriting kinematics analysis. 
Taking under consideration the six axes of innovation derived from the state of the art (see section \ref{sec:axes}), and the needs arising for a more accurate, available and personalized clinical practice, we based our approach on the following qualities:

\begin{enumerate}
\item Cost efficiency. The solutions we propose are low cost.
\item Availability. We mostly utilized ubiquitous devices and well established methods.
\item Ease of use. The tools we built require no particular expertise to handle and are familiar to most people.
\item Quality of diagnosis. Through clinical trials and statistical analysis we validated our methods and proved high correlation with the clinical practice mainstay.
\item Effective monitoring. The solutions we designed allow for long term monitoring through data persistence.
\item Biobanking. Our protocols can be easily reproduced and the data can easily bubble up to offer large datasets for more scientists and further analysis. 
\end{enumerate}

In the following chapters we will present the protocols and tools for \gls{PD} symptoms quantification we introduced, and discuss their design details, implementation and implications for clinical practice. 