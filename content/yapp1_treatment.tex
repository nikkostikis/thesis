\chapter{Treatment}
\label{app:appTreatment}
\pagestyle{fancy}
\fancyhf{}
\fancyhead[OC]{\leftmark}
\fancyhead[EC]{\rightmark}
\cfoot{\thepage}

%%%%%%%%%%%%%%%%%%%%%%%%%%%%%%%%%%%%%%%%%%%%%%%%%%%%%%%%%%%
%%%%%%%%%%%%%%%%%%%%%%%%%%%%%%%%%%%%%%%%%%%%%%%%%%%%%%%%%%%

\section{Medication}
\label{sec:medication}
Considering the great uncertainty surrounding diagnosis, combined with the lack of knowledge on PD causality, it is only inevitable that there is a plethora of pharmacological therapies, forcing the physician into an application pattern of trial and error. As mentioned before, the treatment of PD targets the symptoms and not the cause of the disease. Therefore, no matter the regimen, the progression of the disease is not halted. The goal of the medication is to maintain or increase the quality of PD patients' lives. When patients are in the ON-state they have a good response to the medication and are mostly alleviated from the symptoms. During the OFF-state, either the response to the drug is unsatisfactory or the drug has worn off. The desired state is obviously the former, but should always be achieved within the protocol of safe medication, i.e. no excessive or increased dosages that could have adverse side effects. 

%%%%%%%%%%%%%%%%%%%%%%%%%%%%%%%%%%%%%%%%%%%%%%%%%%%%%%%%%%%

\subsection{Levodopa}
\label{subsec:levodopa}
The most effective drug used for early-diagnosed patients is levodopa, a precursor of dopamine combined with chemicals that help the drug cross the blood-brain barrier. Unfortunately, although this drug is very effective in reducing the symptoms of the disease its chronic use can cause motor complications, such as involuntary movements, also called dyskinesias, along with response variations and unpredictable switching between the ON and OFF-state, also called motor fluctuations (National Collaborating Centre for Chronic Conditions, 2006). 

%%%%%%%%%%%%%%%%%%%%%%%%%%%%%%%%%%%%%%%%%%%%%%%%%%%%%%%%%%%

\subsection{Dopamine Agonists}
\label{subsec:dopamineAgonists}
To avoid Levodopa-induced motor anomalies, physicians could opt for other medication with similar effect to levodopa but without the repercussions. Dopamine receptor agonists are an alternative. They mimic the effect of dopamine and have fewer motor complications. Although in the past they were used as a complement to levodopa only in later stages of the disease, they are now considered a valid alternative for early PD, with the intention to delay the need for levodopa (National Collaborating Centre for Chronic Conditions, 2006). 

%%%%%%%%%%%%%%%%%%%%%%%%%%%%%%%%%%%%%%%%%%%%%%%%%%%%%%%%%%%

\subsection{Apomorphine}
\label{subsec:apomorphine}
In extreme cases where PD patients suffer from severe motor fluctuations, with six or seven alternations between ON and OFF-states a day, and are not responsive to changes in oral medication, apomorphine intermittent injections or continuous infusions via a pump could be beneficial. Apomorphine is a dopamine agonist, which has been proven to reduce PD symptoms (Katzenschlager, 2009). It is a morphine derivative but does not actually contain morphine or its skeleton. Oral administration of apomorphine would always be combined with antiemetic medication, as the drug is a potent emetic (i.e. vomiting inducing). Subcutaneous administration is preferred and can be beneficial due to apomorphine's high potency as a dopamine agonist (Chaudhuri and Clough, 1998). However, apomorphine has severe side effects including but not limited to confusion and hallucinations. 

%%%%%%%%%%%%%%%%%%%%%%%%%%%%%%%%%%%%%%%%%%%%%%%%%%%%%%%%%%%

\subsection{MAO-B}
\label{subsec:maob}
Another treatment aiming to delay the need for levodopa could include monoamine oxidase type B inhibitors (MAO-B), which block the metabolism of dopamine, maintaining its levels in the striatum. A Cochrane review (Macleod et al, 2005) found that although MAO-B inhibitors do not delay the progression of PD, they could be used early in the disease as a monotherapy or as an adjunct to other medications. 

%%%%%%%%%%%%%%%%%%%%%%%%%%%%%%%%%%%%%%%%%%%%%%%%%%%%%%%%%%%
%%%%%%%%%%%%%%%%%%%%%%%%%%%%%%%%%%%%%%%%%%%%%%%%%%%%%%%%%%%

\section{Stereotactic Surgery}
\label{sec:surgery}
In PD, medication is used to help patients manage the symptoms. Unfortunately there is no pharmacological cure for the neuronal degeneration the disease causes. Levodopa, dopamine agonists and MAO-B can be used as monotherapies or adjunctively to alleviate PD manifestations. However, long-term use can cause side effects, such as dyskinesia and motor fluctuations, where the patient's OFF-state is longer and more frequent. In those cases, which mostly apply to later stages of the disease, when adjustments in the doses or modifications of the regimens are no longer beneficial, surgery could be an alternative. The advances in imaging technology have allowed surgeons to perform stereotactic surgery on PD patients to address dopaminergic therapy's motor complications. Stereotactic surgery is a delicate and precise method of locating and targeting structures deep in the brain via a three dimensional coordinate system. This type of surgery involves stimulation, lesioning or ablation of the targeted areas (Lozano et al, 2009). For PD the procedure is called Deep Brain Stimulation (DBS) and it entails inserting a neurostimulator, also referred to as brain pacemaker, consisting of one or two electrodes (unilateral or bilateral DBS, depending on PD laterality), also called leads, implanted in the brain, and a battery powered stimulator usually placed under the patient's collar bone. The leads deliver electrical pulses to the targeted area to interfere with its impaired neuronal activity (Sarem-Aslani and Mullett, 2011). An expert can exogenously calibrate the stimulator.
 
Interestingly enough, although widely used for many years, scientists do not know exactly why DBS works well towards alleviating PD symptoms. So far they can only hypothesize (García et al, 2013). To reduce the impact of PD symptoms DBS is used to target the neurons firing at tremor frequency in the thalamus, the STN, or the GPi, in an attempt to better regulate the direct and indirect basal ganglia pathways. DBS is found to reduce the severity of the motor impairments caused by PD and help the patients reduce or even quit the dopaminergic medication. STN stimulation is found to reduce depression related scores, whereas GPi is believed to have fewer side effects in patients' cognitive functionality. Thalamic DBS, while effective, can have far more serious side effects, increasing the risk of strokes and is therefore superseded mainly by STN DBS (National Collaborating Centre for Chronic Conditions, 2006). 