
this is a sample table
\begin{center}
 \begin{tabular}{||c c c c||} 
 \hline
 Col1 & Col2 & Col2 & Col3 \\ [0.5ex] 
 \hline\hline
 1 & 6 & 87837 & 787 \\ 
 \hline
 2 & 7 & 78 & 5415 \\
 \hline
 3 & 545 & 778 & 7507 \\
 \hline
 4 & 545 & 18744 & 7560 \\
 \hline
 5 & 88 & 788 & 6344 \\ [1ex] 
 \hline
\end{tabular}
\end{center}



Find that thing about UPDRS and why it shouldn't correlate well to the handwriting approach. I think it must be in the 1st or 2nd chapter.

Leave no space between units and the value. 

%Many scientists work at the intersection of engineering, the life sciences and health care to provide solutions for problems in medical practice pertaining to the aforementioned axes. They combine mathematics, statistics and physics with applied sciences, like mechanical and computer engineering, and apply them to established medical practice, either to complement it or even to replace old and problematic techniques. Biomedical engineering, bioinformatics and health informatics are interdisciplinary applied sciences that can offer new insight into the human body and present new diagnostic and therapeutic tools.

%More specifically, in the last decade the scientific community has been trying to find ways to assist medical experts and neurologists to overcome the problems of the \gls{PD} diagnostic protocol.


-The authors calculated tremor amplitude, tremor regularity, power distribution, defined as the percentage of power within the 3–7Hz frequency band, median power frequency, peak power frequency, power dispersion, defined as the frequency band containing 68\% of total power centered at the median power frequency, power dispersion centered at peak power frequency, and harmonic index.-

Say something about action-intention tremor? A task in Denaults paper (sitting and keeping arms and hands in front while trying to bring the tips of the fingers as close as possible to each other) assesses that

The following was also "butchered" from the same subsection and it could help you in the following chapters:

\hl{Correlation analysis showed that the smartphone on-board processing yielded similar results when compared to the post-processing of the phone's raw signal with all tasks' correlation coefficients being above 0.90 for all calculated characteristics}. Similarly, the smartphone's on-board processing correlated well with the post-processing of the separate accelerometer's signal, with correlation coefficients ranging from 0.88 to 0.95 for all tasks and characteristics, except for the results from task 4. However the authors deemed the low correlation of the kinetic task based metrics trivial because the goal was to mainly assess abnormal postural and rest tremor. The most valuable features extracted from the acceleration signals were the time domain related ones, namely tremor amplitude and regularity, whereas the most promising frequency domain features were power distribution, median power frequency and harmonic index. These five characteristics had the highest correlation among the three assessment methods. In order to further validate their method Daneault and his associates recruited 16 patients, 12 diagnosed PD patients, 3 with ET and 1 suffering from multiple sclerosis. They performed a correlation analysis between the results from the smartphone acceleration collection and on-board processing, and a custom scale they created to assess the patients clinically. They found a strong relationship between the amplitude of tremor measured and calculated by the device and the one measured with the scale. The smartphone calculated consistently different amplitude mean values corresponding to each level of the scale, indicating that it could in fact be a valid method of creating tremor profiles over time. 
According to the authors, the reasons for creating a custom rating scale and not using the UPDRS were that:\par
\begin{enumerate}
\item The scale needed to provide a quantitative measure of tremor amplitude, 
\item The scale needed to provide a high degree of precision, i.e., small increments between ordinal values
\item The scale needed to be linear.
\end{enumerate}