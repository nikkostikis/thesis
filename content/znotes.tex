
The crude sensitivity of 5-point TRSs is a good reason for supplementing these scales with precision (transducer) measures when possible (Elble et al, 2006)


%%%%%%%%%%%%%%%%%%%    SmartCT2L    %%%%%%%%%%%%%%%%%%%%%%%%%%%%%%%%%%%%%%%%%%%%%%%%%%%%%%%%%%%%%%%

In Table VI we present the percent differences “on”-“off” in the four metrics, along with the UPDRS scores (“off” and “on”) for both PDDN subjects. We would expect those differences to be negative because we expect higher metric scores (more pronounced tremor) while off-medication or in a de novo state, and lower after the drug ingestion (“on”). 
From the UPDRS scores of the two volunteers it is clear that these two patients did not suffer from severe hand tremor. Their physician observed that the medication improved mostly the patients’ rigidity (which is not measured by our tool) and less so their tremor. However, it is encouraging to note that the readings of the smartphone’s sensors respond well and follow the expected negative trend of the changes in the UPDRS scores in the “on” state. The only discrepancy is observed in the eR position of volunteer B for all metrics, however for that position there were also no observed clinical changes in the UPDRS “on”-“off” as well. 


%%%%%%%%%%%%%%%%%%%    JBHI Rationale    %%%%%%%%%%%%%%%%%%%%%%%%%%%%%%%%%%%%%%%%%%%%%%%%%%%%%%%%%%%%%%%

Tremor is not the main quality of life constraining-aspect of the disease [4], but responds well to dopaminergic medication; as such, its objective quantification can provide useful feedback on the efficacy of the treatment regimen. 
When it comes to PD, it is important to be able to assess patients accurately and frequently in order to adjust their treatment as necessary when variations in the severity of the symptoms occur. Unfortunately, in addition to the time and costs involved, it is difficult for most patients (even more so for those in rural areas), to be monitored frequently by a specialized physician. This creates the need for tools that can aid the physician by assessing a patient’s condition remotely, and that are practical and easy to use. 

The goal of this work is to investigate the use of a smartphone-based tool for assessing PD induced hand tremor. Our approach involves using the phone’s embedded accelerometer and gyroscope sensors to quantify PD hand tremor. Operationally, the patient simply visits a web site [14] and takes up simple postures much like those used in standardized clinical exams, with the smartphone mounted on their hand. The data thus recorded can then be used to classify a subject as healthy or not, and to track the severity of the tremor in PD patients. 
This paper substantially differs from and extends our previous work by 
i) including additional samples from healthy subjects which are age-matched to the PD patients used in our study, 
ii) including data on a small sample of patients off medication in order to quantitatively track the severity of their hand tremor, 
iii) exploring the correlation between our quantitative metrics results and the patients’ (subjective) clinical examination for all supported postures, and 
iv) using a machine learning feature classification approach to choose those metrics which are better at distinguishing between pathological and healthy signals, thus  increase our method’s accuracy. 

The proposed method has been implemented in the form of an app and is available for use on any smartphone with iOS or Android installed, without the need for any downloads or memory-consuming installations [14]. It can be offered as a web-service, so that developers can build their own applications around it and extend its functionality. Our approach does not require the presence of an expert or any kind of special equipment to conduct measurements. It transmits data in real time via TCP/IP, connecting the patient to his physician with no delays, and can benefit the research community by providing anonymized information on PD hand tremor profiles. 

%%%%%%%%%%%%%%%%%%%    END - JBHI Rationale    %%%%%%%%%%%%%%%%%%%%%%%%%%%%%%%%%%%%%%%%%%%%%%%%%%


This paper is a continuation of our previous work [3]; its main contribution is a statistical comparison between signal-based methods of quantifying Parkinsonian tremor using a smartphone, and the UPDRS scores assigned by a physician specialist, in order to validate our previous approach. We acquired hand tremor signals from twenty-three patients using an iPhone and computed the correlation (Pearson product-moment) between the metrics under consideration and the patients’ UPDRS scores regarding hand tremor. Our results indicate a strong correlation (r>0.7) with high statistical significance (p<0.01), which suggests that our quantitative methods of measuring Parkinsonian hand tremor [3] show promise as a means of systematically tracking that component of the disease, possibly as part of a clinical exam or in telemedicine applications. 

In the following, we will specify the combination of a patient’s hand (Right of Left upper extremity) during each position as rR, rL, eR, and eL for rest-right, rest-left, extended-right, and extended-left, respectively.

As per the protocol described earlier, the volunteers were asked to maintain certain postures for 30 seconds, during which the application automatically collected the accelerometer and gyroscope data and sent them to our server. We then used the data to extract features which quantify and characterize the subjects’ tremor levels. The signals were sampled at 20Hz, which is sufficient to identify events occurring at 9Hz or less [20], such as PD-induced tremor. 
Our web application, being written in PHP and JavaScript, is entirely independent of the client's hardware or software platform. It only demands basic prerequisites such as an embedded accelerometer and gyroscope and one of the most popular smartphone operating systems, iOS or Android. 







Find that thing about UPDRS and why it shouldn't correlate well to the handwriting approach. I think it must be in the 1st or 2nd chapter.

Leave no space between units and the value. 

%Many scientists work at the intersection of engineering, the life sciences and health care to provide solutions for problems in medical practice pertaining to the aforementioned axes. They combine mathematics, statistics and physics with applied sciences, like mechanical and computer engineering, and apply them to established medical practice, either to complement it or even to replace old and problematic techniques. Biomedical engineering, bioinformatics and health informatics are interdisciplinary applied sciences that can offer new insight into the human body and present new diagnostic and therapeutic tools.

%More specifically, in the last decade the scientific community has been trying to find ways to assist medical experts and neurologists to overcome the problems of the \gls{PD} diagnostic protocol.


-The authors calculated tremor amplitude, tremor regularity, power distribution, defined as the percentage of power within the 3–7Hz frequency band, median power frequency, peak power frequency, power dispersion, defined as the frequency band containing 68\% of total power centered at the median power frequency, power dispersion centered at peak power frequency, and harmonic index.-

Say something about action-intention tremor? A task in Denaults paper (sitting and keeping arms and hands in front while trying to bring the tips of the fingers as close as possible to each other) assesses that

The following was also "butchered" from the same subsection and it could help you in the following chapters:

\hl{Correlation analysis showed that the smartphone on-board processing yielded similar results when compared to the post-processing of the phone's raw signal with all tasks' correlation coefficients being above 0.90 for all calculated characteristics}. Similarly, the smartphone's on-board processing correlated well with the post-processing of the separate accelerometer's signal, with correlation coefficients ranging from 0.88 to 0.95 for all tasks and characteristics, except for the results from task 4. However the authors deemed the low correlation of the kinetic task based metrics trivial because the goal was to mainly assess abnormal postural and rest tremor. The most valuable features extracted from the acceleration signals were the time domain related ones, namely tremor amplitude and regularity, whereas the most promising frequency domain features were power distribution, median power frequency and harmonic index. These five characteristics had the highest correlation among the three assessment methods. In order to further validate their method Daneault and his associates recruited 16 patients, 12 diagnosed PD patients, 3 with ET and 1 suffering from multiple sclerosis. They performed a correlation analysis between the results from the smartphone acceleration collection and on-board processing, and a custom scale they created to assess the patients clinically. They found a strong relationship between the amplitude of tremor measured and calculated by the device and the one measured with the scale. The smartphone calculated consistently different amplitude mean values corresponding to each level of the scale, indicating that it could in fact be a valid method of creating tremor profiles over time. 
According to the authors, the reasons for creating a custom rating scale and not using the UPDRS were that:\par
\begin{enumerate}
\item The scale needed to provide a quantitative measure of tremor amplitude, 
\item The scale needed to provide a high degree of precision, i.e., small increments between ordinal values
\item The scale needed to be linear.
\end{enumerate}





%i) including additional samples from healthy subjects which are age-matched to the PD patients used in our study, ii) including data on a small sample of patients off medication in order to quantitatively track the severity of their hand tremor, iii) exploring the correlation between our quantitative metrics results and the patients’ (subjective) clinical examination for all supported postures, and iv) using a machine learning feature classification approach to choose those metrics which are better at distinguishing between pathological and healthy signals, thus  increase our method’s accuracy. The accuracy in separating healthy from PD subjects attained in this work is on par with other works using smartphones’ accelerometers and short-duration data [12]. It is also very close to the sensitivity and specificity achieved in [17], where the authors used sensors of the SHIMMER platform to perform mobile gait analysis. There are works that achieve near 100\% accuracy but do so using day-long signals [18] which may not be practical in our setting. High accuracy (98.5\% sensitivity, 97.5\% specificity) is also achieved using smartphone accelerometer-based gait analysis, with a combination of tests and metrics [19]. 
%The proposed method has been implemented in the form of a website and is available for use on any smartphone with iOS or Android installed, without the need for any downloads or memory-consuming installations [14]. It can be offered as a web-service, so that developers can build their own applications around it and extend its functionality. Our approach does not require the presence of an expert or any kind of special equipment to conduct measurements. It transmits data in real time via TCP/IP, connecting the patient to his physician with no delays, and can benefit the research community by providing anonymized information on PD hand tremor profiles. 

%%%%%%%%%%%%%%%%%%%%%%%%%%%%%%%%%%%%%%%%%%%%%%%%%%%%%%%%%%%
%%%%%%%%%%%%%%%%%%%%%%%%%%%%%%%%%%%%%%%%%%%%%%%%%%%%%%%%%%%


%%%%%%%%%%%%%%%%%%%%%%%%%%%%%%%%%%%%%%%%%%%%%%%%%%%%%%%%%%%
\subsection{Signal and Metrics}
\label{subsec:smartSignalMetrics}
%For each volunteer’s session we obtained two signals from the phone’s sensors, the acceleration vector 〖α(i)=[α_x (i),α_y (i),α_z (i)]〗^T (in m/s2) and the  rotational velocity vector 〖ω(i)=[ω_x (i),ω_y (i),ω_z (i)]〗^T (in deg/s), with i denoting discrete time. The rotational velocity in practice should be more information-rich because it is constructed using both accelerometer and gyro data and is expected to capture more of the characteristics of the tremor. We applied a band-pass filter with cutoff frequencies of 1.5Hz and 9.5Hz, in order to exclude noise due to breathing, pulse, or any high-frequency sudden movements during the recordings. The spectral analysis of α(i) and ω(i) of a PD volunteer with typical Parkinsonian tremor is shown in figure 2. As expected, her acceleration and rotational velocity signal amplitude peaks at about 3-5Hz, which is consistent with the literature [20]. We used the acceleration and rotational velocity signals as in [16], to compute the following four metrics for each session:
%where: 〖mag〗_α and 〖mag〗_ω are the sums of squared magnitudes of the acceleration, and the rotation rate vector respectively, and 〖sd〗_α, is the sum of absolute differences in the acceleration vector, summed over each of the three axes, x, y, and z. To compute the 〖mAmp〗_ω metric (4) we initially obtained the magnitude of the Fourier transform of each of the three axial components of the rotation vector ω(i), defined as (ω_κ ) ̂(ξ), κ∈{x,y,z}. We then determined each component’s maximum in the 4≦ ξ ≦7 Hz range (that range being consistent with the frequency of Parkinsonian tremor) and summed the three maxima [16]. 

%%%%%%%%%%%%%%%%%%%%%%%%%%%%%%%%%%%%%%%%%%%%%%%%%%%%%%%%%%%
%%%%%%%%%%%%%%%%%%%%%%%%%%%%%%%%%%%%%%%%%%%%%%%%%%%%%%%%%%%
\section{Signal Processing}
\label{sec:smartSignalProcessing}
%%%%%%%%%%%%%%%%%%%%%%%%%%%%%%%%%%%%%%%%%%%%%%%%%%%%%%%%%%%
\subsection{Means Testing}
\label{subsec:smartMeansTesting}
%Since our goal is to facilitate monitoring and diagnosis of PD-induced tremor, it is essential to establish that the metrics described in the previous section can be useful in differentiating the PD vs H populations. We used the non-parametric Mann-Whitney test to establish that the two populations have statistically different means in all four metrics on all four postures, rR, rL, eR, eL. As shown in Table III, all between-groups tests found significant differences between the mean scores of the metrics of the H volunteers compared to the mean of the PD volunteers. This suggests that the two populations (H and PD) have statistically different scores under every one of the signal metrics computed, and one may attempt to differentiate H vs PD subjects based on one or more of those metrics.
% There was no statistical difference in the subjects’ left versus right mean scores within each group (see Table IX in Appendix). It is typical for PD patients to manifest the disease’s symptoms with some laterality, i.e., to a greater degree on one side, right or left. That is indeed the case with our PD volunteers because 19 of the 23 have differences between the sums of the UPDRS components concerning right vs left hand tremor indicating laterality of motor impairment. Although clinically observable, the Mann-Whitney test for the summed UPDRS scores for right vs left hands yields no statistical difference, with p=0.7327. In order to identify the laterality statistically, for each metric we summed the scores of both postures for right and left hand separately and calculated the absolute differences between them. As shown in the last four rows of Table III, for each metric, the absolute differences between hands for the PD group is statistically different from those of the H group. That means that the amount of difference between hands is not the same for PD and H, presumably due to the disease’s laterality.

%%%%%%%%%%%%%%%%%%%%%%%%%%%%%%%%%%%%%%%%%%%%%%%%%%%%%%%%%%%
\subsection{Features Identification}
\label{subsec:smartFeaturesIdentification}
%%%%%%%%%%%%%%%%%%%%%%%%%%%%%%%%%%%%%%%%%%%%%%%%%%%%%%%%%%%
\subsection{Correlation with Scale-based Metrics}
\label{subsec:smartCorrelation}
%In previous work [16] we attempted to establish the validity of our smartphone-based method of upper limb parkinsonian tremor quantification by running a Pearson product-moment correlation analysis between the UPDRS scores of the PD volunteers and their respective signal metrics. Table IV contains the results of the correlation analysis for the Rest posture scores. The numbers are slightly different from our previous study because the signals are now band pass filtered, as previously explained. In [16] we used the Rest posture data only, whereas here we were also interested in the Extension posture data for both hands. The correlation analysis of the Extension posture data (Table V) yields low coefficients (r<0.6) with low confidence (p>0.01). The results are better for the right hand but generally do not suggest good correlation between the UPDRS scores and the smartphone metrics. These findings show a connection between the manifestation of the action hand tremor and the hardware experimental setup. The fact that the resting tremor is identified consistently, whereas in the extended posture the measured tremor correlates weakly with the clinical assessment, is probably related to the effect of the mass of the smartphone on the dynamics of the hand/arm system.
%%%%%%%%%%%%%%%%%%%%%%%%%%%%%%%%%%%%%%%%%%%%%%%%%%%%%%%%%%%
%%%%%%%%%%%%%%%%%%%%%%%%%%%%%%%%%%%%%%%%%%%%%%%%%%%%%%%%%%%
\section{Machine Learning Model}
\label{sec:smartMachineLearning}

%%%%%%%%%%%%%%%%%%%%%%%%%%%%%%%%%%%%%%%%%%%%%%%%%%%%%%%%%%%
%%%%%%%%%%%%%%%%%%%%%%%%%%%%%%%%%%%%%%%%%%%%%%%%%%%%%%%%%%%
\section{Discussion}
\label{sec:smartDiscussion}

%%%%%%%%%%%%%%%%%%%%%%%%%%%%%%%%%%%%%%%%%%%%%%%%%%%%%%%%%%%
\subsection{OFF - State Trials}
\label{subsec:smartOffState}
%The 23 volunteers of the PD group who underwent the smartphone-based tremor measuring procedure were under medication, but the timespan from their last dose of L-DOPA was anywhere from 1 to 4 hours when they were tested. That means that some of them were at the peak of the drug’s effect while for others this was not the case. With an eye towards tracking the progression of PD, we wanted to see how our metrics would “react” to an alteration in a patient’s condition, such as that brought on by medication intake. We observed two PD volunteers, referred to as the PDDN group, who stayed in the clinic overnight and followed our experimental protocol both “off” and “on” medication (i.e., right before taking their medication in the morning and one hour after that). In Table VI we present the percent differences “on”-“off” in the four metrics, along with the UPDRS scores (“off” and “on”) for both PDDN subjects. We would expect those differences to be negative because we expect higher metric scores (more pronounced tremor) while off-medication or in a de novo state, and lower after the drug ingestion (“on”). From the UPDRS scores of the two volunteers it is clear that these two patients did not suffer from severe hand tremor. Their physician observed that the medication improved mostly the patients’ rigidity (which is not measured by our tool) and less so their tremor. However, it is encouraging to note that the readings of the smartphone’s sensors respond well and follow the expected negative trend of the changes in the UPDRS scores in the “on” state. The only discrepancy is observed in the eR position of volunteer B for all metrics, however for that position there were also no observed clinical changes in the UPDRS “on”-“off” as well.


