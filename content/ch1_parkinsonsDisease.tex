\chapter{Parkinson's Disease}
\label{ch:pd}
\pagestyle{fancy}
\fancyhf{}
\fancyhead[OC]{\leftmark}
\fancyhead[EC]{\rightmark}
\cfoot{\thepage}

%%%%%%%%%%%%%%%%%%%%%%%%%%%%%%%%%%%%%%%%%%%%%%%%%%%%%%%%%%%
%%%%%%%%%%%%%%%%%%%%%%%%%%%%%%%%%%%%%%%%%%%%%%%%%%%%%%%%%%%

\section{General Characteristics}
\label{sec:generalCharacteristics}
\subsection{Demographics}
\label{subsec:demographics}

Parkinson's disease (\gls{PD}) is one of the most representative and frequently encountered chronic movement disorders, affecting more than 1\% of people over 55 and more than 3\% of those over 75 years of age (Van Den Eeden et al, 2003). It is the second most common neurodegenerative disorder after Alzheimer's disease (Pullman and Saunders-Pullman, 2001). People in developed countries are more likely to be affected than those in under-developed or developing ones. This however may be due to differences in life expectancy, rates of late diagnosis of the disease and poor quality of diagnostic services offered (Okubadejo et al, 2010). 

\hlpink{Place a pie here referring to demographics}

%%%%%%%%%%%%%%%%%%%%%%%%%%%%%%%%%%%%%%%%%%%%%%%%%%%%%%%%%%%

\subsection{Manifestations and Pathophysiology}
\label{subsec:manifestations}
PD itself is caused by low and constant falling secretion of dopamine in the substantia nigra pars compacta (\gls{SNc}), an area located in the basal ganglia (\gls{BG}) of the midbrain, and is believed to be responsible for regulating the functionality of the striatum, which is the subcortical part of the forebrain. \gls{BG} is responsible, among other things, for motor coordination, both promoting and inhibiting movement as necessary to produce smooth and fluid motion patterns (Pioli et al, 2008). As we will discuss in a later chapter, although the manifestations and characteristics of \gls{PD} have been heavily researched and documented, including both motor and non-motor anomalies, the actual \textit{reason} why the dopaminergic neurons in the \gls{SNc} die, resulting in abnormally low dopamine levels in the midbrain, is unknown. Thus current treatments only focus on alleviating the symptoms and improving the patients' daily life, rather than targeting what causes the disease.

As described in (Jankovic, 2008) the cardinal motor features establishing a \gls{PD} diagnosis are mainly tremor, rigidity, akinesia or bradykinesia, and loss of postural reflexes and instability, grouped under the acronym \gls{TRAP} (for Tremor, Rigidity, Akinesia and Postural instability). Secondary motor manifestations could include hypomimia, dysarthria, dysphagia, sialorrhoea, micrographia, shuffling gait, festination, freezing, dystonia and glabellar reflexes. The most predominant non-motor symptoms are impaired cognitive and neurobehavioral function, executive dysfunction, bradyphrenia and memory problems. These may be coupled with sleep disorders and sensory abnormalities such as anosmia, paresthesias and pain.

The diagnosis of \gls{PD} is not an easy task. It requires an experienced and well-trained medical professional to distinguish between Parkinson's and other diseases with similar symptoms. It is generally the \textit{absence} of an important manifestation such as rest tremor combined with poor response to specific medication that helps physicians eliminate the possibility of \gls{PD} and attribute the relevant symptoms to other disorders such as essential tremor, or even Alzheimer's disease (Sagar, 1987; Alvarez et al, 2007).

When \gls{PD} manifestations initially occur, they are usually mild and the impact on patients' lives is small. However, it is shown that by the time symptoms are noticeable, the patient has already suffered a greater than 80\% decrease in the dopamine secretion from the \gls{SNc} to the striatum (Langston, 1989). \gls{PD} being a degenerative disease inherently means that without proper treatment all \gls{TRAP} and secondary symptoms are bound to get worse. Cognitive decline comes at a later stage but is often the dominant disabling factor for a patient. 

%%%%%%%%%%%%%%%%%%%%%%%%%%%%%%%%%%%%%%%%%%%%%%%%%%%%%%%%%%%

\subsection{Laterality}
\label{subsec:laterality}
One of the cardinal characteristics of \gls{PD}, which is in fact used to exclude other disorders and bias the diagnosis towards \gls{PD}, is laterality. Most structures of the brain involved in the pathophysiology of the disease's manifestations have left and right sides, controlling contralateral parts of the body. For example, the left side of the SNc affects movements of the right arm (Benarroch, 2006). In \gls{PD}, particularly in the early stages, one side of the brain structures is affected the most, leading to the symptoms appearing unilaterally, i.e., on one side of the patient's body. The side of the body mostly affected on \gls{PD} onset defines the disease's laterality (left or right). 

%%%%%%%%%%%%%%%%%%%%%%%%%%%%%%%%%%%%%%%%%%%%%%%%%%%%%%%%%%%

\subsection{Mortality}
\label{subsec:mortality}
Despite the multiple and some times severe symptoms of \gls{PD}, people die \textit{with}, not \textit{because} of it. It is not considered fatal, but patients have a shorter life expectancy than the general population. The work of (Forsaa et al, 2010) analyzed data from 211 \gls{PD} patients for more than 15 years; some key findings were that:
\begin{itemize}
\item The average time from the first occurrence of \gls{PD} motor symptoms to death was 16 years. 
\item The average age at death was 81.
\item Patients with dementia were twice as likely to die early as patients not suffering from memory problems.
\item The risk of early death increased by 40\% for every 10-year increase in age at diagnosis.
\end{itemize}
The last finding shows that the later the diagnosis, the more likely a patient is to die younger. \gls{PD} does not cause patients' deaths per se, but \gls{PD} symptoms can contribute to fatal incidents. For example, food aspiration into the lungs can be caused by dysphagia, leading to pneumonia or other pulmonary conditions. Falls and even more so recurrent ones, happening at a frequency of at least twice a year, affect more than 50\% of \gls{PD} patients (Allen et al, 2012). Frequent falls could potentially cause serious injuries or even death (Iwasaki et al, 1990; Morgante et al, 2000). While one would instinctively assume that rigidity and postural instability are reasons enough for patients to suffer from frequent falls, studies show correlation between falls and other \gls{PD} manifestations such as executive cognitive impairment, autonomic dysfunction and sleep disturbances (Schrag et al, 2015). This last study concluded that falls are contributed to a number of factors that extend beyond motor impairment, but are associated with \gls{PD}. 

%%%%%%%%%%%%%%%%%%%%%%%%%%%%%%%%%%%%%%%%%%%%%%%%%%%%%%%%%%%
%%%%%%%%%%%%%%%%%%%%%%%%%%%%%%%%%%%%%%%%%%%%%%%%%%%%%%%%%%%

\section{The TRAP Symptom Complex}
\label{sec:trap}
\subsection{Tremor}
\label{subsec:tremor}
Among \gls{PD} motor-related symptoms, the most widely recognized even by non-physicians is involuntary tremor. It is also one of the earliest manifestations of the disease. It affects both upper and lower limbs, and the facial muscular system, mainly the jaw. Tremor in general is defined as the involuntary, rhythmic oscillation of a body part within a fixed plane, involving alternating or simultaneous contractions of agonist and antagonist muscles (Chen and Swope, 2003), or ``any approximately rhythmic, and roughly sinusoidal movement'' (Fahn, 2011). Tremor is primarily measured in terms of its frequency and amplitude. The types of tremors are resting and action, with the latter breaking into more categories depending on the manifestation context (Nowak et al, 2012). More specifically, depending on under which circumstances it occurs, tremor can be classified as:
\begin{itemize}
\item Rest, when the limb is placed at rest on a surface. It usually ceases to exist when voluntary movement is initialized.
\item Action, observed with the application of any kind of force. It can be:
\begin{itemize}
\item Postural, when the limb is simply resisting the forces of gravity.
\item Isometric, when the limb is constantly applying force to hold or push against an object.
\item Kinetic, when there is movement and can be:
\begin{itemize}
\item Simple, when there is no goal.
\item Task-specific, when there is a task to complete.
\item Intention, when there is a goal to be achieved at the end of the movement. Intention tremor, particularly occurs when a limb approaches the endpoint of deliberate and visually guided movement and is usually perpendicular to the axis of the movement.
\end{itemize}
\end{itemize}
\end{itemize}



(\textcolor{BurntOrange}{Table X}).

%% Table will be placed here

In \gls{PD} the most common symptom is a ``pill-rolling'' resting tremor of the hand at a frequency of 4 to 6Hz. This resting tremor often coexists with a postural tremor, i.e., tremor occurring when a person maintains a position against gravity, at frequencies ranging from 4 to 10Hz. The postural tremor in \gls{PD} is usually re-emergent in the sense that, on arm extension for example, a mean latency of 5 seconds occurs before the tremor expresses itself (Chen and Swope, 2003). In the early stages of the disease tremor is usually unilateral. With the progression of the disorder the symptom can manifest bilaterally.

At a pathophysiological level, \gls{PD}-related tremors are induced by neuronal oscillations in central parts of the brain. More specifically, well coordinated voluntary movements are selected and triggered by the basal ganglia (\gls{BG}) operating in conjunction with the cerebellum (Schnitzler and Gross, 2005; Moroney et al, 2008). The basal ganglia is a cluster of neural structures and forms a loop with the motor cortex and the thalamus. In this loop information flows are controlled by neuronal discharges, i.e., neurons firing. There are three modes of neuronal discharge, irregular, bursting and oscillatory. Oscillatory discharge in general leads to neuronal synchronization between various loops of parts of the brain, and is essential in the brain's activity. However, abnormal synchronization processes have been associated with neuropsychiatric disorders and \gls{PD} (Schnitzler and Gross, 2005). Brown discusses the argument that there seems to be a link between the type of neuronal discharge both in the subthalamic nucleus (\gls{STN}) and the globus pallidus internal (\gls{GPi}) located in the \gls{BG}, and the tremor in \gls{PD} (Brown 2003). STN is thought to facilitate movement, whereas GPi plays an inhibitory role. The neuronal discharge in the \gls{STN} is controlled by dopamine levels in the striatum. The dopamine in the striatum is controlled by the SNc. In \gls{PD} patients the secretion of dopamine from the \gls{SNc} to the striatum is low and falling, leaving the striatum with low levels of the neurotransmitter. This results in faulty regulation of the \gls{GPi} and the \gls{STN} by the striatum and increased neuronal firing and enhanced oscillatory and synchronized activity in the \gls{STN}. This long-range synchronization is of abnormal pattern and usually occurs at the frequency of parkinsonian rest and action tremor, at 3 - 10Hz. The oscillations are transmitted from the basal ganglia to the motor cortex by means of the thalamus. From the motor cortex the oscillations are translated to tremor in the body.
 
The procedure described above pertains to the basal ganglia direct and indirect pathways, which are the directions in which the cells in the striatum (striatal cells) project. The indirect pathway is responsible for inhibiting movement, i.e., for preventing involuntary muscle activation by reducing the activation of the thalamus, whereas the direct pathway is responsible for promoting movement by allowing the activation of the thalamus. The abnormal synchronization between the \gls{STN} and the \gls{GPi} is responsible for reducing the inhibition of activation of the thalamus that would result in normal patterns and finally in the over-excitation of the cortex, producing tremor. 

\hlpink{Maybe an illustration here?}

%%%%%%%%%%%%%%%%%%%%%%%%%%%%%%%%%%%%%%%%%%%%%%%%%%%%%%%%%%%

\subsection{Rigidity}
\label{subsec:rigidity}
Rigidity is defined as the involuntary resistance presented in a joint as a response to passive movement, i.e., movement not generated by the patient but forced upon him exogenously. It is usually accompanied by the ``cogwheel'' phenomenon, where the joint feels like it contains cogwheels, resulting in a jerky resistance to an exogenously triggered passive movement. It may occur in all joints, regardless of their proximity to the torso. Rigidity is in most cases exacerbated when the contralateral limb is performing voluntary activation, known as the Froment's maneuver (Broussolle et al, 2007).

The pathophysiology of rigidity in \gls{PD} is a matter of debate (Santens et al, 2003). Scientists have not found any direct relationship between dopaminergic denervation and rigidity, constituting the known \gls{PD} related deficiencies in the basal ganglia-thalamocortical loop insufficient to explain this \gls{PD} symptom. The Froment's maneuver phenomenon implies that the pathophysiology of rigidity involves a wider, distributed brain loop (Baradaran et al, 2013). Overall, it is believed that no single mechanism is responsible for parkinsonian rigidity (Xia, 2011).

Interestingly, one of the most frequent manifestations of \gls{PD} usually present at disease onset is shoulder pain, which could possibly be associated with rigidity (Jankovic, 2008).


%%%%%%%%%%%%%%%%%%%%%%%%%%%%%%%%%%%%%%%%%%%%%%%%%%%%%%%%%%%

\subsection{Akinesia, Bradykinesia and Hypokinesia}
\label{subsec:akinesia}
Bradykinesia, akinesia and hypokinesia are three terms describing similar \gls{PD} characteristics, and although they are usually considered synonyms they define contextually different manifestations. All terms are of Greek origin. Strictly speaking, Bradykinesia defines slowness in the execution of a voluntary movement; akinesia refers to involuntary ``freezing'' prior to the initiation of movement and is often used to characterize the lack of arm swing during walking; hypokinesia describes the execution of smaller than desired movements to complete a task, as it happens for example in patients' micrographia during handwriting. Although these three symptoms are related, they should be assessed separately, since they are not always well correlated with each other in individual patients (Berardelli et al, 2001).

The most important form of akinesia is ``freezing''. \gls{PD} patients whose main symptom is not tremor more frequently suffer from freezing of gait (Macht et al, 2007), also called ``motor blocks''. These patients exhibit hesitation to start walking, turning, walking in tight quarters, reaching their destination and walking in open space. The disrupted messages from the basal ganglia to the motor cortex affect \gls{PD} patients' ability to perform complex automated and combined motions. Patterns autonomously coordinated and executed effortlessly by healthy individuals, require extensive brain activation for \gls{PD} patients (Péchadre et al, 1976). This results in the freezing of gait, identified as a few seconds' pause from the desire to execute to the actual initiation of movement.

As we have already mentioned, it is believed that abnormal oscillatory activity of the neurons in the basal ganglia at the \gls{PD} tremor rate (\textless10Hz) translates in tremor in the body (Brown, 2003; Schnitzler and Gross, 2005). The same studies show that there also exist oscillatory activities in higher frequencies in the basal ganglia, namely at 15 to 30Hz, which correlate strongly with the sparsity and slowness of movement in \gls{PD}. As described in (Boraud et al, 2005), the inverse relationship between 15 to 30Hz oscillatory activity and healthy motor function suggests that the information processing necessary for voluntary movement planning and generation is inhibited by the synchronization of activity in this band. However, oscillations in the loop involving the basal ganglia are not enough to explain the bradykinesia, akinesia and hypokinesia a \gls{PD} patient will experience. In fact, (Boraud et al, 2005) claim that, as is the case with rigidity, a more complex model in which dopamine depletion in some way leads to loss of normal function of the basal ganglia, be it information processing or motor control, needs to be proposed in order to determine the causative influence of striatal dopaminergic denervation to the bradykinetic \gls{PD} symptoms. Although prominent in \gls{PD} patients' basal ganglia loop, synchronized oscillations at 15-30Hz are most likely not a de novo feature of parkinsonism. 


%%%%%%%%%%%%%%%%%%%%%%%%%%%%%%%%%%%%%%%%%%%%%%%%%%%%%%%%%%%

\subsection{Postural Deformities and Instability}
\label{subsec:instability}
In the later stages of \gls{PD}, many patients experience postural deformities and instability. Axial postural rigidity, defined as rigidity of the muscular structures supporting the neck and trunk, does occur and results in abnormal axial postures. Camptocormia, which is defined by extreme truncal flexion, neck flexion, flexed elbows and knees and scoliosis, are among \gls{PD} manifestations (Ashour and Jankovic, 2006).

Striatal limb deformities (e.g. striatal hand, striatal toe) are also common \gls{PD} symptoms (Winkler et al, 2002; Jankovic, 2008). About 23\% of the \gls{PD} patients participating in a study (Winkler et al, 2002) suffered from striatal toe (extension of the big toe), whereas according to  (Kumar et al, 2013) extensor plantar response (i.e. striatal toe) affects 10\% of \gls{PD} patients. These conditions are called striatal because they are believed to be caused by dysfunction of the striatum, but their pathophysiology is unknown.

It is interesting to note that in (Ashour and Jankovic, 2006), the side of striatal deformity was highly correlated with the laterality of initial parkinsonian symptoms in all patients (100\%) with striatal hand and in 83.3\% of patients with striatal foot . Another interesting fact is that striatal deformities are late to appear in \gls{PD} and when there are no other symptoms they lead the physicians to consider alternative diagnosis, such as arthritis or orthopaedic problems. However, they can be early manifestations as well, with lack of the other cardinal issues, such as tremor or bradykinesia (Ashour et al, 2005).

Apart from deformities, \gls{PD} patients suffer from postural instability. The pull test, in which the physician applies force on the patient's shoulders pulling him backward or forward abruptly, is used to assess the degree of retropulsion (backward pull) or propulsion (forward pull). Taking more than two steps to balance off or not responding to the pull at all is considered abnormal postural behavior. It is common for patients with many years past disease onset to demonstrate abnormal response to the force applied by taking more than two steps to counteract the pull or push, or by failing to move in a timely fashion. 